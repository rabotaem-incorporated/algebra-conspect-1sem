\subsection{Свойства степени}

\begin{theorem-non}
    $f, g \in R[x],~\deg f = m,~\deg g = n$

    \begin{enumerate}
        \item $\deg(f + g) \leq \max(m, n)$
        
        При этом: $m \neq n \implies \deg(f + g) = \max(m, n)$

        
        \item $\deg(fg) \leq m + n$
    \end{enumerate}
\end{theorem-non}

\begin{proof}
    \begin{enumerate}
        \item $f = \sum\limits_{i = 0}^m a_i X^i,~g = \sum\limits_{i = 0}^n b_i X^i,~d = \max(m, n)$
        
        $f = \sum\limits_{i = 0}^d a_i X^i,~g = \sum\limits_{i = 0}^d b_i X^i$

        $f + g = \sum\limits_{i = 0}^d (a_i + b_i) X^i \implies \deg(f + g) \leq d$

        $m \neq n \implies \begin{cases}
            a_d = 0\\
            b_d \neq 0
        \end{cases}$ или $\begin{cases}
            a_d \neq 0\\
            b_d = 0
        \end{cases} \implies a_d + b_d \neq 0 \implies \deg(f + g) = d$


        \item $\left( \sum\limits_{i = 0}^m a_iX^i \right) \left( \sum\limits_{j = 0}^n b_jX^j \right) = a_0b_0 + (a_0b_1 + a_1b_0)X + \ldots + a_mb_nX^{m+n} \implies \deg fg \leq m + n$
    \end{enumerate}
\end{proof}

\begin{notice}
    $\deg fg < m + n$, если $a_m \neq 0$ или $b_n \neq 0$ и $a_m b_n = 0$
\end{notice}

\begin{notice}
    Будем считать, что $\deg 0 = -\infty$
\end{notice}

\begin{defn}
    Область целостности (целостное кольцо, область) --- коммутативное ассоциативное кольцо с $1 \neq 0$ и без делителей нуля.

    \[ a \neq 0 \text{ так чтобы } \exists b \neq 0 : ab = 0 \]
\end{defn}

\begin{theorem-non}
    Пусть $R$ --- ОЦ(область целостности). 

    \begin{enumerate}
        \item $\forall f,g \in R[x]: \deg(fg) \leq \deg f + \deg g$
        
        \item $R[x]$ --- ОЦ
    \end{enumerate}
\end{theorem-non}

\begin{proof}
    \begin{enumerate}
        \item В предыдущем доказательстве $\begin{cases}
            a_m \neq 0\\
            b_n \neq 0
        \end{cases} \implies a_m b_n \neq 0 \implies \deg(fg) = m + n$

        \item $f \neq 0 \implies \deg f \geq 0,~g \neq 0 \implies \deg g \geq 0 \implies \deg(fg) \geq 0 \implies fg \neq 0$
    \end{enumerate}
\end{proof}

\begin{follow}
    Пусть $R$ --- ОЦ: тогда $R[x]^* = R^*$
\end{follow}

\begin{proof}
    Очевидно $R^* \subset R[x]^*$

    Обратно, пусть $f \in R[x]^* \implies$

    $\exists g \in R[x]: f \cdot g = 1 (\implies f, g \neq 0)$

    $\deg(fg) = 0 = \deg f + \deg g \implies \deg f = \deg g = 0 \implies f  \in R^*$
\end{proof}

\begin{examples}~

    \begin{enumerate}
        \item $\Z[x]^* = \{\pm 1\}$
        
        \item $\R[x]^* = \R \setminus \{0\}$
    
        \item $(\Z/4\Z)[x]^*$ --- бесконечное множество
    \end{enumerate}
\end{examples}

\begin{exerc}
    $R[[x]]^* = \{ \sum\limits_{i = 0}^{\infty} a_iX^i \mid a_0 \in R^* \}$    
\end{exerc}
