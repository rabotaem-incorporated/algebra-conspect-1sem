\subsection{Алгебраически замкнутые поля. Каноническое разложение над \CC и над \R.}

\begin{defn}
    Поле $K$ называется \emph{алгебраически замкнутым}, если любой $f \in K[x]$ имеет корень в $K$.
\end{defn}

\begin{theorem}
    Основная теорема алгебры.

    $\CC$ алгебраически замкнуто.
\end{theorem}

\begin{proof}
     Не будет в курсе.

     Идея доказательства:

     $f = a_nx^n + \dots + a_1x + a_0,~ z \in \CC,~ f(z) = 0$.

     $r > max\{|a_0|, \ldots, |a_n|\}$.

     $f(r(\cos(\varphi) + i\sin(\varphi))) = r^n(\cos(n\varphi) + i\sin(n\varphi)) + g(r(\cos(\varphi) + i\sin(\varphi)))$.

     $|g(r(\cos(\varphi) + i\sin(\varphi)))| < r^{n - 1}(|a_{n - 1}| + \ldots + |a_1| + |a_0|) < r^n$.

     $\implies \Delta \arg f(r(\cos(\varphi) + i\sin(\varphi))) = 2\pi n$.

     $D = \{z \in \CC \mid |z| \le r\}$

     $\overset{\text{Топология}}{\implies} f(D)$ - односвязная область.

     $\implies 0 \in f(D) \implies \exists z~f(z) = 0$.
\end{proof}

\begin{notice} 
    Любое поле можно вложить в алгебраически замкнутое поле. 
    
    Всегда есть минимальное такое поле.

    Для $\Q$ это поле алгебраических чисел.

    Алгебраическое число - комплексный корень многочлена над $\Q$.
\end{notice}

\begin{theorem-non}
    $K$ - алгебраически замкнутое поле, $f \in K[x]$.

    Тогда $f$ - неприводим $\iff \deg f = 1$.
\end{theorem-non}

\begin{proof}
    Все многочлены $\deg = 1$ неприводимы.

    $\deg f \neq 1 \implies \exists x \in K: f(x) = 0$

    $\overset{\text{Т. Безу}}{\implies} (x - c) \mid f$

    Таким образомб если $f \in K[x], \deg f \ge 1$, то его каноническое разложение имеет вид:

    $f = c_0\prod\limits_{i = 1}^n (x - c_i)^{d_i}$, где $c_i \in K,~d_i \in \Z_+$.
\end{proof}

\begin{theorem-non}
    $f \in \R[x],~ a \in CC$ - его корень.

    Тогда $\bar{a}$ - корень $f$ той же кратности.
\end{theorem-non}

\begin{proof}~
    $l$ - кратность $a$.

    В $\CC[x]~f = (x - a)^lg,~ g \in \CC[x],~ g(a) \neq 0$.

    Пусть $g = b_nx^n + \dots + b_1x + b_0$.

    Рассмотрим $\bar{g} = \bar{b_n}x^n + \dots + \bar{b_1}x + \bar{b_0}$.

    Тогда $f = \bar{f} = \overline{(x - a)^l}\bar{g} = (x - \bar{a})^l\bar{g} \implies f(\bar{a}) = 0$

    $g(a) = \overline{\bar{g}(\bar{a})} \implies x - \bar{a} \nmid \bar{g}$

    $\implies \bar{a}$ - корень $f$ кратности $l$

    $\implies$ все корни входят парами

    $\implies f = r_0\left(\prod\limits_{i = 1}^n (x - r_i)^{d_i}\right) \cdot \left(\prod\limits_{i = 1}^m ((x - c_i)(x - \bar{c_i}))^{p_i}\right)$, где $r_i \in \R,~d_i \in Z_+,~c_i \in \CC,~p_i \in \Z_+$.

    $\implies f = r_0\left(\prod\limits_{i = 1}^n (x - r_i)^{d_i}\right) \cdot \left(\prod\limits_{i = 1}^m B_i^{p_i}\right),~ B_i$ - квадратичные многочлены, неприводимые в $\R$.
\end{proof}

\begin{theorem-non}
    Унитарные неприводимые многочлены в $\R$ - это:

    1. $x - a,~ a \in \R$

    2. $x^2 + ax + b,~ a, b \in \R,~ b^2 - 4ac < 0$
\end{theorem-non}

\begin{proof}~
    С многочленами степени 1 и 2 все ясно.

    Если степень многочлены больше 2, то справедливо разложение $9.2$, значит они приводимы.
\end{proof}


