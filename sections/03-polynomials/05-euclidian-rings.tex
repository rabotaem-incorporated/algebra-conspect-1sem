\subsection{Евклидовы области}

\begin{defn}
    \emph{Евклидовой областью целостности} (\emph{евклидовой областью}, \emph{евклидовым кольцом}) называется область целостности $R$, 
    для которой существует функция $\nu: R \setminus \{0\} \to \Z_{\geq 0}$, называемая \emph{евклидовой нормой}, такая что:

    \begin{enumerate}
        \item Если $d \mid a$, то $\nu(d) \leq \nu(a)$, причем $\nu(d) = \nu(a) \iff d \sim a$.
        \item Для любых $a, b \in R,~b \neq 0$: существует представление $a = bq + r$, где $r = 0$ или $\nu(r) < \nu(b)$.
    \end{enumerate}

    \emph{Замечание}: свойство один можно убрать, но доказательства будут сложнее.
\end{defn}

\begin{examples}~

    \begin{enumerate}
        \item $R = K[x]$, ($K$ - поле), где $\nu(P) = \deg P$

        \item $R = \Z$, где $\nu(a) = |a|$
        
        \item $R = \Z[i] = \{ a + bi \mid a,b \in \Z \}$, где $\nu(a + bi) = a^2 + b^2$ 
        (подробнее в книжке Аейрленд, Роузен - <<Классическое введение в современную теорию чисел>>)
        
        \item $R = K[[x]]$, ($K$ - поле)
        
        $R^* = \{ a_0 + a_1 x + a_2 x + \ldots \mid a_0 \neq 0 \}$

        $\ord f =  \min \{ j \mid a_j \neq 0 \}$

        $f = x^{\ord f} \cdot (a_j + a_{j + 1}x + \ldots) \sim x^{\ord f}$ 

        \begin{exerc}
            Докажите, что это евклидова область.
        \end{exerc}

        \item $R = \Z_{(5)} = \{ \frac{a}{b} \mid a, b \in \Z,~5 \nmid b \}$
        
        \begin{exerc}
            Докажите, что это евклидова область.
        \end{exerc}
    \end{enumerate}
\end{examples}

\begin{lemma}
    Пусть $R$ - область целостности, $a, b \in R$. Тогда $a \sim b \iff a = \eps b,~\eps \in R^*$
\end{lemma}

\begin{proof}

    <<$\Longleftarrow$>>:

    Пусть $a = \eps b \implies b$. Так как $\eps$ --- обратим, $b = \eps^{-1} a$. 
    \begin{gather*}
        \begin{rcases}
            a = \eps b \implies a \mid b & \\
            b = \eps^{-1} a \implies b \mid a &
        \end{rcases} \iff a \sim b        
    \end{gather*}
    <<$\Longrightarrow$>>: 
    \begin{gather*}
        a \sim b \implies \begin{cases}
            a \mid b\\
            b \mid a
        \end{cases} \implies \begin{cases}
            b = \eps a\\
            a = \eps' b
        \end{cases} \implies b = \eps \eps' b \implies (\eps \eps' - 1) b = 0 \implies \eps \text{\ --- обратим}.
    \end{gather*}
\end{proof}

\begin{defn}    
    $R$ --- коммутативное кольцо, $I \subset R$ называется \emph{идеалом} в $R$, если:

    \begin{enumerate}
        \item $I \neq \emptyset$
        
        \item $\forall a, b \in I: a + b \in I$
        
        \item $\forall a \in I~\forall b \in R: ab \in I$
    \end{enumerate}
\end{defn}

\begin{examples}~

    \begin{enumerate}
        \item $R = \Z,~I = 2\Z$
        
        \item $R = K[X],~I = \{ f \in R \mid f(0) = 0 \}$
        
        \item $R = C[0, 1]$\ (непрерывные функции на отрезке [0, 1]), $~I = \{ f \in R \mid f(0) = 0 \}$
    \end{enumerate}
\end{examples}

\begin{defn}
    Пусть $R$ --- коммутативное кольцо, $r \in R$.
    Из свойств кольца очевидно, что $\langle r \rangle \rightleftharpoons  (r) \rightleftharpoons \{ rs \mid s \in R \}$ --- идеал в $R$.

    Тогда $(r)$ называется \emph{главным идеалом} порожденный элементом $r$.
\end{defn}

\begin{notice}
    $(r) = (r') \iff r \sim r'$
\end{notice}

\begin{example}~
    Пример неглавного идеала:
    \begin{gather*}
        R = \Z[X],~I = \{ f : 2 \mid f(0) \}
    \end{gather*}
\end{example}

\begin{theorem-non}
    В евклидовой области все идеалы главные.
\end{theorem-non}

\begin{proof}
    Пусть $I$ - идеал в области целостности $R$.

    Случай $I = \{0\}$ --- тривиален, тогда $I = (0)$. Пусть $I \neq \{0\}$.

    Зафиксируем норму $\nu$ и рассмотрим $c \in I$ с минимальной нормой. Докажем, что $I = (с)$.

    <<$\supset$>>:
    
    Так как для любого $b \in R$ должно быть выполнено $cb \in I$, то $I \supset (c)$.
    
    <<$\subset$>>:

    Предположим, $\exists a \in I \setminus (c)$. Представим евклидову норму в виде $a = cq + r$, $q, r \in R$.
    Если $r = 0$, то $a \in (c)$ по определению главного идеала. Но иначе $\nu(r) < \nu(c)$.
    Выразим $r$:
    \begin{gather*}
        r = a - cq = a + c(-q).
    \end{gather*} 
    Так как $c \in I$ и $a \in I$, то и $c(-q) \in I$, следовательно $r \in I$.
    Но $\nu(r) < \nu(c)$, что противоречит минимальности нормы $\nu(c)$
\end{proof}

\begin{defn}   
    Область целостности, в которых все идеалы главные, называется \emph{областью главных идеалов (ОГИ)}.
\end{defn}

\begin{theorem-non}
    Пусть $R$ --- область главных идеалов. Тогда:

    \begin{enumerate}
        \item $a, b \in R \implies$ у $a$ и $b$ существует наибольший общий делитель
        
        \item Если $d$ - наибольший общий делитель $a$ и $b$, то $d = am + bn,~m, n \in R$
    \end{enumerate}
\end{theorem-non}

\begin{proof}
    
    Можно считать $a \neq 0$ или $b \neq 0$, если $a = b = 0$, то $d = 0$ подходит, $d \neq 0$ не подходит.

    \begin{enumerate}
        \item Рассмотрим множество $I = \{ am + bn \mid m, n \in R \}$ --- идеал в $R$. Тогда можно записать $I = (d)$. 
        
        Заметим, что $d$ --- общий делитель $a$ и $b$. 
        \begin{gather*}
            \begin{cases}
                a = a \cdot 1 + b \cdot 0 \in I = (d)\\
                b = a \cdot 0 + b \cdot 1 \in I = (d)
            \end{cases} \implies \begin{cases}
                d \mid a\\
                d \mid b
            \end{cases}
        \end{gather*}
        Покажем, что $d$ - наибольший общий делитель $a$ и $b$.
        То есть, что 
        \begin{gather*}
            \begin{cases}
                d' \mid a\\
                d' \mid b
            \end{cases} \lbl{$(!)$}{\implies} d' \mid d
        \end{gather*}
        Так как $d \in I$, $d = a m_0 + b n_0,~m_0, n_0 \in R$.
        \begin{gather*}
            \begin{cases}
                d' \mid a\\
                d' \mid b
            \end{cases} \implies \begin{cases}
                d' \mid a m_0\\
                d' \mid b n_0
            \end{cases} \implies d' \mid d.
        \end{gather*}
        Что и требовалось доказать.

        \item Если $d'$ - наибольший общий делитель $a$ и $b$, то:
        \begin{gather*}
            d' \sim d \in I \implies d' \in I \implies d' = am + bn,~m, n \in R.
        \end{gather*}
    \end{enumerate}
\end{proof}

\begin{notice}
    Наибольший общий делитель в ОГИ обозначают $(a, b)$.
\end{notice}

\begin{defn}
    Элементы  ОГИ $a$ и $b$ называют \emph{взаимно простыми}, если $(a, b) = 1$.
\end{defn}

\begin{theorem-non}
    $(a, b) = 1 \iff m, n \in R: am + bn = 1$
\end{theorem-non}

\begin{proof}
    <<$\Longrightarrow$>>: Из предыдущего предложения

    <<$\Longleftarrow$>>: $d = (a, b) \implies d \mid a, d \mid b \implies d = 1 \implies d \sim 1$
\end{proof}
