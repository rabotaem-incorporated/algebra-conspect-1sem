\subsection{Алгебраически замкнутые поля. Каноническое разложение над \CC и над \R.}

\begin{defn}
    Поле $K$ называется \emph{алгебраически замкнутым}, если любой $f \in K[x]$ имеет корень в $K$.
\end{defn}

\begin{theorem}
    Основная теорема алгебры.

    $\CC$ алгебраически замкнуто.
\end{theorem}

\begin{proof}
     Не будет в курсе.

     Идея доказательства:

     $f = a_nx^n + \dots + a_1x + a_0,~ z \in \CC,~ f(z) = 0$.

     $r > max\{|a_0|, \ldots, |a_n|\}$.

     $f(r(\cos(\phi) + i\sin(\phi))) = r^n(\cos(n\phi) + i\sin(n\phi)) + g(r(\cos(\phi) + i\sin(\phi)))$.

     $|g(r(\cos(\phi) + i\sin(\phi)))| < r^{n - 1}(|a_{n - 1}| + \ldots + |a_1| + |a_0|) < r^n$.

     $\implies \Delta \arg f(r(\cos(\phi) + i\sin(\phi))) = 2\pi n$.

     $D = \{z \in \CC \mid |z| \leq r\}$

     $\overset{\text{Топология}}{\implies} f(D)$ --- односвязная область.

     $\implies 0 \in f(D) \implies \exists z~f(z) = 0$.
\end{proof}

\begin{notice} 
    Любое поле можно вложить в алгебраически замкнутое поле. 
    
    Всегда есть минимальное такое поле.

    Для $\Q$ это поле алгебраических чисел.

    Алгебраическое число --- комплексный корень многочлена над $\Q$.
\end{notice}

\begin{theorem-non}
    $K$ --- алгебраически замкнутое поле, $f \in K[x]$.

    Тогда $f$ --- неприводим $\iff \deg f = 1$.
\end{theorem-non}

\begin{proof}
    Все многочлены степени $1$ неприводимы.

    $\deg f \neq 1 \implies \exists x \in K: f(x) = 0$

    $\overset{\text{Т. Безу}}{\implies} (x - c) \mid f \implies$ он приводим

    Таким образом если $f \in K[x],~ \deg f \geq 1$, то его каноническое разложение имеет вид: \[f = c_0\prod\limits_{i = 1}^n (x - c_i)^{d_i},\] где $c_i \in K,~d_i \in \Z_+$.
\end{proof}

\begin{theorem-non}
    $f \in \R[x],~ a \in \CC$ --- его корень.

    Тогда $\oln{a}$ --- корень $f$ той же кратности.
\end{theorem-non}

\begin{proof}
    
    Пусть $l$ --- кратность корня $a$.

    В $\CC[x]$ имеем $f = (x - a)^lg,\quad g \in \CC[x],~ g(a) \neq 0$.

    Пусть $g = b_nx^n + \dots + b_1x + b_0$.

    Рассмотрим $\oln{g} = \oln{b_n}x^n + \dots + \oln{b_1}x + \oln{b_0}$.

    Тогда $f = \oln{f} = \oln{(x - a)^l}\oln{g} = (x - \oln{a})^l\oln{g} \implies f(\oln{a}) = 0$

    $0 \neq g(a) = \overline{\oln{g}(\oln{a})} \implies (x - \oln{a}) \nmid \oln{g}$

    $\implies \oln{a}$ --- корень $f$ кратности $l$

    $\implies$ все корни разбиваются на пары сопряженных, тогда каноническое разложение в $\CC[x]$ имеет вид:
    \[f = r_0\left(\prod\limits_{i = 1}^n (x - r_i)^{d_i}\right) \cdot \left(\prod\limits_{i = 1}^m ((x - c_i)(x - \oln{c_i}))^{p_i}\right),\]
    где $r_i \in \R,~d_i \in Z_+,~c_i,\oln{c_i} \in \CC,~p_i \in \Z_+$.
    \[\iff f = r_0\left(\prod\limits_{i = 1}^n (x - r_i)^{d_i}\right) \cdot \left(\prod\limits_{i = 1}^m B_i^{p_i}\right),\] 
    $B_i$ --- квадратичные многочлены, неприводимые в $\R$.

    $B_i = (x - c_i)(x - \oln{c_i}) = x^2 - (c_i + \oln{c_i})x + c_i \oln{c_i} = x^2 - 2 \Re c_i x + |c_i|^2 \in \R[x]$
\end{proof}

\begin{theorem-non}
    Унитарные неприводимые многочлены в $\R$ --- это:

    1. $x - a,\quad a \in \R$

    2. $x^2 + ax + b,\quad a, b \in \R,~ b^2 - 4ac < 0$
\end{theorem-non}

\begin{proof}~
    С многочленами степени 1 и 2 все ясно.

    Если степень многочлена больше 2, то справедливо разложение $9.2$, значит он приводим.
\end{proof}


