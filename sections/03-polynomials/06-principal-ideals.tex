\subsection{Факториальность области главных идеалов}

\begin{defn}
    Пусть $R$ - коммутативное кольцо.
    Элемент $a$ называется \emph{неприводимым}, если $a \neq 0$, $a \notin R^*$ и $a = bc \implies b \in R:*$ или $c \in R^*$.

    То есть неприводимый элемент --- необратимый элемент, который не раскладывается в произведение двух обратимых. 
\end{defn}

\begin{defn}
    \emph{Приводимый элемент} --- не $0$, не обратный, не неприводимый.
\end{defn}

\begin{examples}~

    \begin{enumerate}
        \item $R = K[x]$, ($K$ - поле)
        
        $\deg f = 1 \implies f$ - неприводимый, так как $f = bc, \deg f = \deg b + \deg c = 1 + 0 = 0 + 1 = 1$

        \item $\R[x]$: 
        
        $x^2 - 4$ приводим $x^2 - 4 = (x - 2)(x + 2)$
    
        $x^2 + 1$ неприводим, иначе имел бы корень в $\R$
    \end{enumerate}
\end{examples}

\begin{lemma}
    
    Пусть $f \in K[x]$ --- многочлен степени $2$ или $3$. Тогда $f$ приводим $\iff$ у него есть корень в $K$.
\end{lemma}

\begin{proof}

    <<$\Longrightarrow$>>:

    $a$ - корень $f \lbl{т. Безу}{\implies} (X - a) \mid f$. Рассмотрим разложение $f = (X - a) \cdot g$.
    Так как $\deg g = \deg f - 1 \geq 1$, оно нетривиально и $f$ --- приводимый.

    <<$\Longleftarrow$>>:

    Пусть $f = gh$ и $\deg g,~\deg h \geq 1$, не умаляя общности, $\deg g \geq \deg h$. Тогда:
    \begin{gather*}
        \underbrace{\deg f}_{2\text{ или }3} = \deg g + \deg h
    \end{gather*}
    Есть два стула: $2 = 1 + 1$ и $3 = 2 + 1$ (на какой сам сядешь, на какой друга посадишь?)
    \begin{gather*}
        \deg h = 1 \implies h = aX + b \implies h \left( -\frac{b}{a} \right) = 0 \implies f \left( -\frac{b}{a} \right) = 0.
    \end{gather*}
    Значит $-\frac{b}{a}$ --- корень $f$.
\end{proof}

\begin{notice}
    Многочлены большей степени могут быть приводимыми, но не иметь корней в $K$.
    \begin{example}
        Рассмотрим $f = x^4 + x^2 + 1$ в $\R[x]$:
        \begin{gather*}
            f = (x^2 + 1)^2 = (x^2 + x + 1)(x^2 - x + 1).
        \end{gather*}
    \end{example}
\end{notice}
