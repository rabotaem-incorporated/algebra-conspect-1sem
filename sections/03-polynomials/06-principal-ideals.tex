\subsection{Факториальность области главных идеалов}

\begin{defn}
    Пусть $R$ --- коммутативное кольцо.
    Элемент $a$ называется \emph{неприводимым}, если $a \neq 0$, $a \notin R^*$ и $a = bc \implies b \in R^*$ или $c \in R^*$.

    То есть неприводимый элемент --- необратимый элемент, который не раскладывается в произведение двух обратимых. 
\end{defn}

\begin{defn}
    \emph{Приводимый элемент} --- элемент, который не является ни $0$, ни обратимым, ни неприводимым.
\end{defn}

\begin{examples}~

    \begin{enumerate}
        \item $R = K[x]$, ($K$ --- поле)
        
        $\deg f = 1 \implies f$ --- неприводимый, так как $f = bc,~ \deg f = \deg b + \deg c = 1 + 0 = 0 + 1 = 1$

        \item $\R[x]$: 
        
        $x^2 - 4$ приводим $x^2 - 4 = (x - 2)(x + 2)$
    
        $x^2 + 1$ неприводим, иначе имел бы корень в $\R$
    \end{enumerate}
\end{examples}

\begin{lemma}
    
    Пусть $f \in K[x]$ --- многочлен степени $2$ или $3$. Тогда $f$ приводим $\iff$ у него есть корень в $K$.
\end{lemma}

\begin{proof}

    <<$\Rightarrow$>>:

    $a$ --- корень $f \lbl{т. Безу}{\implies} (X - a) \mid f$. Рассмотрим разложение $f = (X - a) \cdot g$.
    Так как $\deg g = \deg f - 1 \geq 1$, оно нетривиально и $f$ --- приводимый.

    <<$\Leftarrow$>>:

    Пусть $f = gh$ и $\deg g,~\deg h \geq 1$, не умаляя общности, $\deg g \geq \deg h$. Тогда:
    \begin{gather*}
        \underbrace{\deg f}_{2\text{ или }3} = \deg g + \deg h
    \end{gather*}
    Есть два стула: $2 = 1 + 1$ и $3 = 2 + 1$ (на какой сам сядешь, на какой друга посадишь?)

    В любом случае:
    \begin{gather*}
        \deg h = 1 \implies h = aX + b \implies h \left( -\frac{b}{a} \right) = 0 \implies f \left( -\frac{b}{a} \right) = 0.
    \end{gather*}
    Значит $-\frac{b}{a}$ --- корень $f$.
\end{proof}

\begin{notice}
    Многочлены большей степени могут быть приводимыми, но не иметь корней в $K$.
    \begin{example}
        Рассмотрим $f = x^4 + 2x^2 + 1$ в $\R[x]$:
        \begin{gather*}
            f = (x^2 + 1)^2 = (x^2 + x + 1)(x^2 - x + 1).
        \end{gather*}
    \end{example}
\end{notice}

\begin{notice}
    Далее считается, что $R$ --- область главных идеалов
\end{notice}

\begin{lemma}
    Пусть $p, f \in R$. $p$ --- неприводимый элемент. Тогда $p \mid f$ либо $(p, f) = 1$.
\end{lemma}

\begin{proof}
    $(p, f) \mid p \implies (p, f) = 1 $ или $(p, f) = p \implies (p, f) = 1$ или $p \mid f$.
\end{proof}

\begin{theorem-non}
    Пусть $p$ --- неприводимый$,~ p \mid ab \implies p \mid a$ или $p \mid b$.
\end{theorem-non}

\begin{proof}
    Пусть $p \nmid a,~ p \nmid b \implies (p, a) = (p, b) = 1 \implies$

    $pm + an = 1,~ pm' + bn' = 1 \lbl{перемножим}{\implies} p(pmm' + mbn' + anm') + abnn' = 1 \implies p \mid 1$
\end{proof}

\begin{defn}
    Область целостности $R$ называют факториальным кольцом, если:

    \begin{enumerate}
        \item Любой $a \in R$ отличный от $0$ и не являющийся обратимым можно представить в виде $a = p_1 \ldots p_s,~ s \geq 1$ и $p_1, \ldots, p_s$ --- неприводимые элементы.
        
        \item Если $p_1 \ldots p_s = q_1 \ldots q_t$, где все $p_i, q_i$ --- неприводимые элементы, то $s = t$ и после перенумерации $q_j$ выполнено $q_1 \sim p_1, \ldots, q_s \sim p_s$.
    \end{enumerate}
\end{defn}

\begin{theorem}
    Область главных идеалов является факториальным кольцом.
\end{theorem}

\begin{proof}

    \begin{enumerate}
        \item <<Существование разложения>>
        
        Есть элемент $a$, докажем что существует неприводимый $p$ такой, что $p \mid a$.

        Возьмем $a$, если он неприводимый, то доказывать нечего (т.к. $a \mid a$), иначе: $a = a_1 b_1$, где $a_1, b_1 \notin R^*$ ($a$ не неприводим $\implies a_1, b_1$ оба обратимы либо оба необратимы, если оба обратимы, то $a$ обратим, а нас такие элементы не интересуют) 
    
        $a_1$ -- неприводимый, следовательно утверждение доказано, иначе $a_1 = a_2 b_2$, где $a_2, b_2 \notin R^*$
    
        и так далее
    
        Предположим, утверждение неверно. Обозначим $I = \bigcup\limits_{i = 1}^{\infty} (a_i)$. 
    
        Можно заметить, что  $a_2 \mid a_1,~ a_3 \mid a_2,~ \ldots \implies (a_1) \subset (a_2) \subset \ldots$
    
        Покажем, что $I$ --- главный идеал. 
        \begin{enumerate}
            \item $x, y \in (a_j) \subset I \implies x+y \in (a_j) \subset I$
    
            \item $x \in I \implies x \in (a_i)$ для некоторого $i$, тогда для $a \in R \implies ax \in (a_i) \subset I$
        \end{enumerate} 
    
        Значит $I$ главный идеал, тогда $I = (c)$ для некоторого $c \in I \implies c \in (a_i)$ для некоторого $i$, при этом $a_{i+1} \in I \implies c \mid a_{i+1}$
    
        $\begin{cases}
            a_i \mid c\\
            c \mid a_{i + 1}
        \end{cases} \implies a_i \mid a_{i+1} \lbl{$a_{i + 1} \mid a_i$}{\implies} a_i \sim a_{i + 1} \implies$ цепочка когда-то прервется, так как $a_i = a_{i + 1} \cdot b_{i + 1}$, где $a_i = \eps a_{i + 1},~ \eps \in R^* \implies b_{i + 1} = \eps \in R^*$ 
    
        Значит любой необратимый элемент делится на неприводимый.

        \item <<Единственность разложения>>

        Пусть $p_1 \ldots p_s = q_1 \ldots q_t$, где все $p_i, q_i$ --- неприводимы и не умаляя общности $s \leq t$
    
        Индукция по $S$
    
        <<База>>: $s = 1$ 
    
        $p_1 = q_1 \ldots q_t$, где $p_1$ --- неприводимый $\implies t = 1,~q_1 = p_1$.
    
        <<Переход>>: $s > 1$
    
        $p_s \mid (p_1 \ldots p_s) \implies p_s \mid (q_1 \ldots q_t) \implies \exists j: p_s \mid q_j$
    
        Перенумеруем, так чтобы $j = t$, тогда $q_t = p_s \cdot \eps$, но $q_t$ неприводим $\implies \eps \in R^*$
        
        $p_1 \ldots p_s = q_1 \ldots q_{t - 1} \cdot \eps p_s \implies p_1 \ldots p_{s-1} = q_1 \ldots q_{t - 1} \cdot \eps = \underbrace{(\eps q_1)}_{\text{неприводимый}} \cdot q_2 \ldots q_{t - 1}$
    
        По индукционному предположению $p_1 \ldots p_{s-1}$ совпадает с $(\eps q_1) \cdot q_2 \ldots q_{t - 1}$ c точностью до порядка и ассоциировнности.
    \end{enumerate}
\end{proof}

\begin{notice}
    Евклидова область $\subset$ Область главных идеалов $\subset$ Факториальное кольцо $\subset$ Область целостности
\end{notice}

\begin{examples}~

    \begin{enumerate}
        \item $R$ --- факториальное кольцо $\implies R[X]$ --- факториальное кольцо.
        
        \item $\Z[X]$ --- факториальное кольцо.
        
        \item $K[X][Y] = K[X, Y] \implies K[X, Y]$ --- факториальное кольцо.
    \end{enumerate}
\end{examples}