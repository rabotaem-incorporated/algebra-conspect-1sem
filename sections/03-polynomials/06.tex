\subsection{Факториальность области главных идеалов}

\begin{defn}
    $a \in R$, $R$ - коммутативное кольцо, называется неприводимым, если $a \neq 0, \oln{a} \notin R^*$ и $a = bc \implies b \in R*$ или $c \in R^*$
\end{defn}

\begin{defn}
    Приводимый элемент --- не $0$, не обратный, не неприводимый.
\end{defn}

\begin{examples}~

    \begin{enumerate}
        \item $R = K[x]$, ($K$ - поле)
        
        $\deg f = 1 \implies f$ - неприводимый, так как $f = bc, \deg f = \deg b + \deg c = 1 + 0 = 0 + 1 = 1$

        \item $\R[x]$: 
        
        $x^2 - 4$ приводим $x^2 - 4 = (x - 2)(x + 2)$
    
        $x^2 + 1$ неприводим, иначе имел бы корень в $\R$
    \end{enumerate}
\end{examples}

\begin{lemma}
    Пусть $f \in K[x]$ --- многочлен степени $2$ или $3$.

    Тогда $f$ приводим $\iff$ у него есть корень в $K$.
\end{lemma}

\begin{defn}

    "$\Rightarrow$":

    $a$ - корень $f \lbl{\text{т.Безу}}{\implies} (X - a) \mid f$

    $f = (X - a) \cdot g$ --- нетривиальное разложение

    $\deg g = \deg f - 1 \geq 1$

    "$\Leftarrow$":

    $f = gh$ и $\deg g,~\deg h \geq 1$, не умаляя общности, $\deg g \geq \deg h$

    $\underbrace{\deg f}_{2\text{ или }3} = \deg g + \deg h$

    Есть два стула: $2 = 1 + 1$ и $3 = 2 + 1$ (на какой сам сядешь, на какой друга посадишь?)

    $\deg h = 1 \implies h = aX + b \implies h \left( -\frac{b}{a} \right) = 0 \implies f \left( -\frac{b}{a} \right) = 0$

\end{defn}

\begin{notice}
    Многочлены большей степени могут быть приводимыми, но не иметь корней в $K$.
\end{notice}

\begin{example}
    $f = x^4 + x^2 + 1$ в $\R[x]$

    $f = (x^2 + 1)^2 = (x^2 + x + 1)(x^2 - x + 1)$
\end{example}