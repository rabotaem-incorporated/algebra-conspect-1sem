\subsection{Деление с остатком}

\begin{theorem}[о делении с остатком для многочленов]

    $R$ --- область целостности.
    
    Пусть $f, g \in R[x],~g \neq 0$ и старший коэффициент $g$ обратим.

    Тогда $\exists!~q, r \in R[x]$:

    \begin{enumerate}
        \item $f = gq + r$
    
        \item $\deg r < \deg g$
    \end{enumerate}
\end{theorem}

\begin{proof}
    Пусть $\deg g = d,~g = b_dX^d + \ldots$

    \begin{enumerate}
        \item <<Существование>>
        
        Индукция по $\deg f:\ \deg f < d \implies$ подходит $q = 0, r = f$

        Пусть $\deg f = n \geq d$

        $f_1 = f - g \cdot a_n \cdot b_d^{-1} \cdot X^{n-d}$, где $b_d$ --- старший коэффициент $g$ (на первый взгляд здесь написано что-то неочевидное, но на деле это простое деление многочленов столбиком, то есть мы просто делаем так, чтобы старший коэффициент $f$ исчез)

        $g \cdot a_n \cdot b_d^{-1} \cdot X^{n-d} = (b_d X^d + \ldots) \cdot a_n \cdot b_d^{-1} \cdot X^{n-d} = a_nX^n + \ldots \implies \deg f_1 < n$

        По индукционному предположению $\exists q_1, r_1 \in R[x]$ такие, что:
        \begin{enumerate}
            \item $f_1 = gq_1 + r_1$
            \item $\deg r_1 < d$
        \end{enumerate}

        $f = g \cdot a_n \cdot b_d^{-1} \cdot X^{n - d} + f_1 = g \underbrace{(a_n \cdot b_d^{-1} \cdot X^{n - d} + q_1)}_{q} + \underbrace{r_1}_{r}$


        \item <<Единственность>>
        
        Предположим $f = g \cdot q_1 + r_1 = g \cdot q_2 + r_2,~\deg r_1 < d,~ \deg r_2 < d$ 
        
        $g(q_1 - q_2) = r_2 - r_1$

        Предположим $q_1 \neq q_2 \implies \deg g \cdot (q_1 - q_2) \lbl{$R$ --- ОЦ}{=} \underbrace{\deg g}_{d} + \underbrace{\deg{q_1 - q_2}}_{\geq 0} \geq d \implies \deg(r_2 - r_1) \geq d$, но $\deg(r_2 - r_1) < d$, противоречие.
    \end{enumerate}
\end{proof}

\begin{notice}
    Условие $R$ --- область целостности не существенно.

    (я без понятия что написано дальше, но пускай будет)

    $g = b_d X^d + \ldots,~b_d \in R^*$

    $b_d \cdot a = 0 \implies b_d^{-1}(b_d a) = 0 \implies a = 0$ (что это значит?)
\end{notice}
