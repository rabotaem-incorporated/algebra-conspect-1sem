\subsection{Рациональные дроби}

\begin{defn}
    Пусть $R$ область целостности. Мы вложим $R$ в поле $Q(R)$, назовем её полем частных.    
\end{defn}

Рассмотрим $M = R \times (R \setminus \{0\})$ и введём на $M$ отношение $\sim$:

$(a, b) \sim (a', b') \iff ab' = a'b$

Проверим: $\sim$ --- отношение эквивалентности

рефлексивность и симметричность очевидны

транзитивность:
$\begin{cases}
    (a, b) \sim (a', b')\\
    (a', b') \sim (a'', b'')
\end{cases} \implies ab'b'' = a'bb'' = a''bb' \implies b'(ab'' - a''b) = 0$

$b \neq 0 \implies ab'' - a''b \implies ab'' = a''b \implies (a, b) \sim (a'', b'')$

То есть $\sim$ --- это отношение эквивалентности на $M$.

$Q(R) = M / \sim = \{ [(a, b)] \mid a \in R, b \in R \setminus \{0\} \}$

\begin{defn}
    Обозначим $\frac{a}{b}$ --- это $[(a, b)] \in Q(R)$.
\end{defn}

Введём в $Q(R)$ операции сложения и умножения:

\begin{gather*}    
    \frac{a_1}{b_1} + \frac{a_2}{b_2} = \frac{a_1b_2 + a_2b_1}{b_1b_2}\\
    \frac{a_1}{b_1} \cdot \frac{a_2}{b_2} = \frac{a_1a_2}{b_1b_2}
\end{gather*}

\begin{notice}
    $(a, b) \sim (ac, bc) \quad \forall c \in R \setminus \{0\}$

    Такая замена не изменит результат.
\end{notice}

\begin{notice}
    $(a, b) \sim (a', b') \iff (ab', bb') = (a'b, b'b)$
\end{notice}

\begin{theorem-non}
    Операции на $Q(R)$ определены корректно, при этом $Q(R)$ с этими операциями --- поле.
\end{theorem-non}

\begin{proof}
    Коммутативность и ассоциативность сложения очевидны в случае одинакого знаменателя.

    $\frac{a_1}{b} + \frac{a_2}{b} = \frac{a_1 b + a_2 b}{b^2} = \frac{a_2 b + a_1 b}{b^2} = \frac{a_2}{b} + \frac{a_1}{b}$

    Но любые $2$ дроби можно привести к общему знаменателю:

    $\frac{a_1}{b_1} = \frac{a_1 b_2}{b_1 b_2} \quad \frac{a_2}{b_2} = \frac{a_2 b_1}{b_1 b_2}$

    Нейтральный по сложению элемент --- это $\frac{0}{1}$

    Противоположный по сложению элемент к $\frac{a}{b}$ --- это $\frac{-a}{b}$

    Дистрибутивность: $\left( \frac{a_1}{b} + \frac{a_2}{b} \right) \frac{a'}{b'} = \frac{a_1 + a_2}{b} \frac{a'}{b'} = \frac{(a_1 + a_2) a'}{b b'} = \frac{a_1a' + a_2a'}{b b'} = \frac{a_1}{b} \frac{a'}{b'} + \frac{a_2}{b} \frac{a'}{b'} = \frac{a_1}{b} \frac{a'}{b'} + \frac{a_2}{b} \frac{a'}{b'}$

    Нейтральный по умножению элемент --- это $\frac{1}{1}$

    $\frac{a}{b} \neq \frac{0}{1} \implies a \cdot 1 \neq b \cdot 0 \iff a \neq 0$

    Обратный по умножению элемент к $\frac{a}{b}$ --- это $\frac{b}{a}$
\end{proof}

$R \lbl{$\eps$}{\mapsto} Q(R),~ r \mapsto \frac{r}{1}$

То есть, считаем $R \subset Q(R)$

\begin{defn}
    Пусть $K$-поле. Тогда поле $K(x) = Q(K[x])$ назовем полем рациональных дробей (дробно-рациональных функций) над полем $K$.
\end{defn}

\begin{theorem-non}[Несократимое представление]
    Пусть $R$ --- факториальное кольцо. Тогда любой $S \in Q(R)$ представимых в виде $s = \frac{p}{q},~(p, q) = 1$. Такое представление единственно с точностью до умножения $p$ и $q$ на $\eps \in R^*$.
\end{theorem-non}

\begin{proof}
    $s = \frac{a}{b},~ d = \gcd(a, b) \implies a = d a', b = d b' \implies s = \frac{a'}{b'},~ (a', b') = 1$

    $\frac{p}{q} = \frac{p'}{q'}, (p, q) = (p', q') = 1$

    $\begin{cases}
        p \mid (p'q)\\
        (p, q) = 1
    \end{cases} \implies p \mid p'$, аналогично $p' \mid p$

    То есть $p$ и $p'$ ассоциативны $\implies$ $p' = \eps p,~ \eps \in R^*$

    $pq' = \eps p q \implies q' = \eps q$
\end{proof}

\begin{lemma}
    Пусть $s \in K(x),~s = \frac{p}{q}, p, q \in K[x]$

    Тогда $\deg p - \deg q$ --- инвариант $s$.
\end{lemma}

\begin{proof}
    $\frac{p}{q} = \frac{p'}{q'} \implies pq' = p'q \implies \deg p + \deg q' = \deg p' + \deg q \implies \deg p - \deg q = \deg p' - \deg q'$

    $\deg \frac{p}{q} = \deg p - \deg q$
\end{proof}

\begin{defn}
    $s \in K(x)$ называется правильной дробью, если $\deg s < 0$

    В частности $0$ --- правильная дробь
\end{defn}

\begin{notice}
    Очевидно сумма и произведение правильных дробей --- правильная дробь
\end{notice}

\begin{lemma}
    Любая рациональная дробь однозначно представляется в виде суммы многочленов и правильной дроби.
\end{lemma}

\begin{proof}
    $s = \frac{p}{q}$

    $p = q \cdot t + r$, где $r = 0$ или $\deg r < \deg q$

    $\implies \frac{p}{q} = t + \frac{r}{q},~t \in K[x],~\frac{r}{q}$ --- правильная дробь

    $t + \frac{r}{q} = t_1 + \frac{r_1}{q_1}, t_1 \in K[x], \frac{r_1}{q_1}$ --- правильная дробь

    $t - t_1 = \frac{r_1}{q_1} - \frac{r}{q}$ --- правильная дробь

    $t - t_1 = \frac{t - t_1}{1} \implies \deg(t - t_1) < 0 \implies  t - t_1 = 0$
\end{proof}

\begin{lemma}
    Пусть $(f, g) = 1$. Тогда любую дробь со знаменателем $fg$ можно представить как сумму дробей со знаменателем $f$ и $g$.
\end{lemma}

\begin{proof}
    $1 = cf + dg$ для некоторых $c, d \in K[x]$

    $\frac{a}{fg} = \frac{a(cd + dg)}{fg} = \frac{acf}{fg} + \frac{adg}{fg} = \frac{ac}{g} + \frac{ad}{f}$
\end{proof}

\begin{proof}
    Дробь $s$ называется примарной ($p$-примарной), если $s = \frac{a}{p^n}$, $p$ --- неприводимый многочлен, $n \in \N$
\end{proof}

\begin{theorem-non}
    Любую правильную дробь можно однозначно представить в виде суммы нескольких отличных от $0$ правильных $p$-примарных дробей, где $p$-различные унитарные неприводимые многочлены.
\end{theorem-non}

\begin{proof}
    <<Существование>>:

    Запишем значение $s$ в виде $p_1^{m_1} \ldots p_t^{m_t}$, где $p_i$ --- унитарные неприводимые многочлены

    $p_1^{m_1} \ldots p_t^{m_t} = (p_1^{m_1} \ldots p_{t-1}^{m_{t-1}}) \cdot p_t^{m_t}$ (старшие коэффиценты ушли в числитель)

    По лемме

    $s = \frac{\ldots}{p_1^{m_1}} + \frac{\ldots}{p_t^{m_t}} = \frac{a_1}{p_1^{m_1}} + \ldots + \frac{a_t}{p_t^{m_t}}$ (многочлен и правильная дробь, с тем же знаменателем)

    $\implies s = f + \frac{b_1}{p_1^{m_1}} + \ldots + \frac{b_t}{p_t^{m_t}},~f \in K[x],~\frac{b_j}{p_j^{m_j}}$ --- правильная дробь

    $\implies f = s - \frac{b_1}{p_1^{m_1}} - \ldots - \frac{b_t}{p_t^{m_t}}$

    $\implies \frac{f}{1}$ --- правильная дробь

    $\implies \deg f < 0 \implies f = 0$

    <<Единственность>>:
    
    Пусть у $S$ есть $2$ различных таких разложения

    Вычитаем из первого разложения второе, получим:

    $\frac{c_1}{p_1^{n_1}} + \ldots + \frac{c_l}{p_l^{n_l}} = 0,~p_i$ --- унитарные неприводимые многочлены

    $c_1, \ldots, c_l \neq 0$

    Можно считать все эти дроби несократимыми.

    $\implies \frac{c_1}{p_1^{n_1}} + \ldots + \frac{c_{l-1}}{p_{l-1}^{n_{l-1}}} = \frac{-c_l}{p_l^{n_l}}$

    Приведем к общему знаменателю и сделаем их сократимыми

    знаменатель будет делиться на $p_1^{n_1} \ldots p_{l-1}^{n_{l-1}}$

    не может быть ассоциированно с $p_l^{n_l}$

    $\frac{b_i}{p_i^{m_i}} - \frac{b_i'}{p_i^{m_i}} = \frac{\ldots}{p_i^{n_i}}$
\end{proof}

\begin{defn}
    Простейшей дробью называется дробь вида $\frac{a}{p^n}$, где $p$ --- неприводимый многочлен, $n \in \N$, $\deg a < \deg p$
\end{defn}

\begin{theorem}
    Любая ненулувая правильная дробь единственным образом представляется в виде суммы нескольких простых дробей с различными знаменателями.
\end{theorem}

\begin{proof}
    <<Существование>>:

    Достаточно разложить правильную примарную дробь $\frac{a}{p^n}$

    $a = r_mp^m + \ldots + r_1p + r_0$, $\deg r_i < \deg p,~r_m \neq 0$

    $m < n$, так как $\deg a < \deg p$

    $\frac{a}{p^n} = \frac{r_m}{p^{n - m}} + \frac{r_{m-1}}{p^{n - m + 1}} + \ldots + \frac{r_1}{p^{n - 1}} + \frac{r_0}{p^n}$ --- искомое представление, если удалить нулевые слагаемые

    <<Единственность>>:

    Пусть для $s$ есть представления в виде суммы примарных. Обозначим $C_p$ за сумму $p$-примарных дробей в этом представлении.

    $C_p$ --- $p$-примарная правильная дробь

    $s = C_{p_1} + \ldots + C_{p_t} \implies$ все $C_p$ определены однозначно
    
    Пусть у $C_p$ есть лва разных разложения в сумму простейших дробей со степенями $p$ в знаменателях.

    $\frac{r_n}{p_n} + \ldots + \frac{r_1}{p_1} = \frac{s_n}{p_n} + \ldots + \frac{s_1}{p_1}$, где $n$ --- максимальная показатель степени $p$ в знаменателях

    Пусть $m$ --- максимальный индекс, такой что $r_m \neq s_m$, некоторые $r_i$ и $s_i$ могут быть нулевыми

    $\frac{r_m - s_m}{p^m} + \ldots + \frac{r_1 - s_1}{p_1} = 0$

    $\implies \frac{r_m - s_m}{p} = -(r_{m-1} - s_{m-1}) - p (r_{m-2} - s_{m-2}) - \ldots \in K[X]$
\end{proof}