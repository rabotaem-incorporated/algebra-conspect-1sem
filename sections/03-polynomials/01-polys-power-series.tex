\subsection{Многочлены и формальные степенные ряды}

\begin{defn}
    Последовательность \emph{финитная} $\iff \exists N: \forall n \geq N: a_n = 0$.
\end{defn}

\begin{defn}
    Многочленом над $R$ (от одной переменной) называется финитная последовательность $(a_i),~a_i \in \R,~i=0,1,2,\ldots$.
\end{defn}

\begin{defn}
    $R$ --- коммутативное кольцо с 1, тогда:
    
    $R[x] = \{(a_i) \mid a_i \in R,~ i = 0, 1, \ldots; a_i = 0$ при $i \to \infty \}$ --- \emph{кольцо многочленов} над $R$.
\end{defn}

\begin{theorem-non}
    Операции в $R[x]$:

    <<Сложение>>:
    $(a_i) + (b_i) = (a_i + b_i)$

    <<Умножение>>:
    $(a_i) \cdot (b_i) = (p_i)$, где $p_k = \sum\limits_{i=0}^k a_i b_{k-i}$
\end{theorem-non}

\begin{theorem-non}[Переход к стандартной записи]~

    $\letus a \in R,~[a] = (a, 0, 0, \ldots)$ --- многочлен, равный $a$.

    $[a] + [b] = [a + b]$
    
    $[a] \cdot [boba] = (aboba, 0, 0, \ldots) = [aboba]$
    
    Отождествим $[a]$ с $a$.
    
    $[a] \cdot (b_0, b_1, \ldots) = (ab_0, ab_1, \ldots)$
    
    $(a_0, a_1, \ldots, a_n, 0, 0, \ldots) = (a_0, 0, 0, \ldots) + (0, a_1, 0, 0, \ldots) + \ldots + (0, 0, \ldots, a_n, 0, 0, \ldots) =$ 
    
    $a_0 \cdot \underbrace{(1, 0, 0, \ldots)}_{x_0} + a_1 \cdot \underbrace{(0, 1, 0, \ldots)}_{x_1} + \ldots + a_n \cdot \underbrace{(0, 0, \ldots, 1, 0, 0, \ldots)}_{x_n} = a_0 + a_1 \cdot x_1 + \ldots + a_n \cdot x_n$
    
    $x_j \cdot x_1 = (0, \ldots, 1, 0, 0, \ldots) \cdot (0, 1, 0, 0, \ldots) = (0, \ldots, 0, 1, 0, 0, \ldots) = x_{j+1} \implies \forall m \in \N: x_m = x_1^m$
    
    $x_1 = x \implies x_m = x_1^m = x^m$
    
    Значит получили стандартную запись многочленов $(a_0 + a_1 x + a_2 x^2 + \ldots + a_n x^n)$
\end{theorem-non}

\begin{defn}
    $\letus f \in R[x],~f \neq 0$ (то есть не $(0)$)

    Тогда степенью $f$ называется максимальное $j$ такое что $a_j \neq 0$

    Обозначим $\deg f = j$.

    Если $f = 0$, то $\deg f \in \{-1, -\infty\}$ (по разному обозначают).
\end{defn}

\begin{defn}
    $d = \deg f \implies a_d$ называется старшим коэффициентом $f$.
\end{defn}

\begin{defn}
    Константой называется множество $f$ такое что $\deg f \leq 0$.
\end{defn}

\begin{defn}
    Мономом называется множество вида $a x^j$.
\end{defn}

\begin{theorem-non}
    $R[x]$ --- коммутативное ассоциативное кольцо с 1.
\end{theorem-non}

\begin{proof}

    \begin{enumerate}
        \item[1-4.] Аксиомы относящиеся к сложению очевидны.
        
        \item[5.] Коммутативность умножения очевидна.
        
        \item[6.] Ассоциативность умножения:
        
        $f, g, h \in R[x],~ (fg)h = f(gh)$

        $f = \sum\limits_{i=0}^k f_i X_i$, где $f_i \in R$

        $g = \sum\limits_{i=0}^l g_i X_i$, где $g_i \in R$

        $h = \sum\limits_{i=0}^n h_i X_i$, где $h_i \in R$
    
        Ассоциативность мономов $(f \cdot g) \cdot h = f \cdot (g \cdot h)$ --- сводится к сложению, $f, g, h$ --- мономы.

        $(fg)h = (\sum f_i X^i \cdot \sum g_j X^j) \cdot \sum h_k X^k = \sum (f_i X^i \cdot g_j X^j) \cdot \sum h_k X^k \lbl{ассоц. мономов}{=}$ 
        
        $\sum f_i X^i \cdot \sum (g_j X^j \cdot h_k X^k) = f(gh)$
        
        \item[7.] Нейтральный элемент по умножению --- $1 = (1, 0, 0, \ldots)$.
        
        \item[8.] Дистрибутивность:
        $\begin{cases}
            (aX^i \cdot bX^j) \cdot cX^k = abX^{i+j} \cdot cX^k = abc \cdot X^{i+j+k}\\
            aX^i \cdot (bX^j \cdot cX^k) = aX^i \cdot bcX^{j + k} = abc \cdot X^{i+j+k}
        \end{cases}$
        
    \end{enumerate}
\end{proof}

\begin{defn}
    $R[[x]] = \{ (a_i) \mid a_i \in R, i = 0, 1, \ldots \}$ --- множество формальных степенных рядов над $R$.
\end{defn}

$(a_i) = \sum\limits_{i = 0}^{\infty} a_i X^i$

\begin{exerc}
    $R[[x]]$ --- коммутативное ассоциативное кольцо с 1.
\end{exerc}