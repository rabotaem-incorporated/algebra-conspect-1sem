\subsection{Формула Тейлора}

\begin{theorem-non}
    $K$ --- поле$,\quad f, g \in K[x],~ f \neq 0,~ d = \deg(g) \geq 1$.

    Тогда $f$ можно представить единственным образом в виде:

    \[f = h_ng^n + \dots + h_1g + h_0,\]

    где $n \geq 0,~ h_i \in K[x],~ h_n \neq 0,~ \deg(h_i) < d,~ \forall i = 0, \ldots, n$.
\end{theorem-non}

\begin{proof}

    <<Существование>>:

    Индукция по $l = \deg f$.

    <<База>>: При $l < d$ подходит $n = 0,~ h_0 = f$.

    <<Переход>>: При $l \geq d:~ f = gq + r,~ \deg(r) < d,~ q \neq 0$.

    $\deg gq \geq \deg g > \deg r$

    $\implies \deg f = \deg gq \implies \deg q = l - d$
    
    По ИП: $q = h_ng^n + \dots + h_1g + h_0,~ h_n \neq 0,~ \deg(h_i) < d,~ i = 0, \dots, n$.

    $\implies f = h_ng^{n + 1} + \dots + h_0g + r$.

    <<Единственность>>:

    Индукция по $l = \deg f$.

    При $l < d$:

    $\deg h_ng^n \geq nd > \deg h_ig^i,~ i = 0, \ldots, n - 1 \implies \deg f = \deg h_ng^n \implies \deg h_ng^n < d$.

    $nd + d - 1 \geq l \geq nd \implies n$ --- неполное частное при делении $l$ на $d$.

    <<База>>: При $l < d:~ n = 0 \implies h_0 = f$.

    <<Переход>>: При $l \geq d$ предположим, что есть еще разложение $f = \hat{h_n}g^n + \dots + \hat{h_1}g + \hat{h_0}$.

    $f = g(h_ng^{n - 1} + \dots + h_1) + h_0 = g(\hat{h_n}g^{n - 1} + \dots + \hat{h_1}) + \hat{h_0},\quad \deg h_0, ~\deg \hat{h_0} < d$

    Тогда $h_0 = \hat{h_0}$. По единственности деления с остатком.

    По ИП: $\deg f_1 = h_ng^{n - 1} + \dots + h_1 < \deg f \implies h_i = \hat{h_i},~ i = 1, \dots, n - 1$.
\end{proof}


\begin{theorem-non}
    $\Char K = 0,~ f \in K[x],~ f \neq 0,~ d = \deg(f) = n \geq 0,~ a \in K$.

    $\implies f = \sum\limits_{i = 0}^n \frac{f^{(i)}(x - a)^i}{i!},~ f^{(i)} \in K[x],~ \deg(f^{(i)}) < d,~ i = 0, \dots, n$.
\end{theorem-non}

\begin{proof}
    $f = \sum\limits_{i = 0}^n c_i(x - a)^i,~ c_i \in K,~ i = 0, \ldots, n,~ c_n \neq 0$.

    $f^{(i)} = \sum\limits_{j = i}^{n} c_{j}j!(x - a)^{j - i},\quad i = 0, \ldots, n$.

    $f^{(j)}(a) = c_jj! \implies c_i = \frac{f^{(i)}(a)}{i!}$.
\end{proof}