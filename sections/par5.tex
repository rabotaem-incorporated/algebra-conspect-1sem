\begin{normalsize}

\section{Наибольший общий делитель}

\begin{defn}
    $R$ --- коммутативное кольцо, $a, b \in R$.\\
    Элемент $d$ называется наибольшим общим делителем, если:\\
    \begin{enumerate}
        \item $d \mid a, d \mid b$
        \item $d' \mid a, d' \mid b \implies d' \mid d$
    \end{enumerate}
\end{defn}

\begin{theorem-non}~
    \begin{enumerate}
        \item $d_1, d_2$ --- наибольшие общие делители, тогда $d_1, d_2$ --- ассоциированны.
        \item $\letus d_1$ --- наибольший общий делитель, $d_2$ ассоциированн с $d_1$, тогда $d_2$ --- тоже наибольший общий делитель.
    \end{enumerate}
\end{theorem-non}

\begin{proof}
    \begin{enumerate}
        \item По свойству 2. $d_1 \mid d_2,~d_2 \mid d_1 \implies d_1,~d_2$ --- ассоциированны.
        \item $d_2 \mid d_1,~d_1 \mid a,~d_1 \mid b \implies d_2 \mid a,~d_2 \mid b$\\
        Пусть $d_2$ не наибольший, тогда $\exists d' > d_2$.\\
        $d' \mid a,~d' \mid b \implies d' \mid d_1$\\
        $d' \mid d_1,~d_1 \mid d_2 \implies d' \mid d_2$\\
        Противоречие
    \end{enumerate}
\end{proof}

\begin{theorem-non} 
    $\letus a, b \in \mathZ \implies$
    \begin{enumerate}
        \item $\exists d \in \mathZ:~a\mathZ + b\mathZ = d\mathZ$
        \item при этом $d$ --- наибольший общий делитель $a, b$.
    \end{enumerate}
\end{theorem-non}

\begin{proof}
    \begin{enumerate}
        \item $I = a\mathZ + b\mathZ$, заметим что $0 \in I$, так как $0a + 0b = 0$.\\
        $I = \{0\} \implies I = 0\mathZ$\\
        $I \neq \{0\} \implies c \in I \implies -c \in I$, так как $-(ax + by) = a \cdot -x + b \cdot -y$\\
        То есть в $I$ есть положительные числа.\\
        $d = \min\{ c \mid c \in I, c > 0 \}$, докажем что $a\mathZ + b\mathZ = d\mathZ$\\
        "$\subset$":\\
        $d \in I \implies d = ax_0 + by_0, x_0, y_0 \in \mathZ \implies$\\
        $\forall z \in \mathZ: dz = a(x_0z) + b(y_0z) \in I$\\
        "$\supset$":\\
        $\letus c \in I, d \in \mathN \implies \exists q, r \in \mathZ: c = dq + r, 0 \leq r < d$\\
        $c = ax_1 + by_1, x_1, y_1 \in \mathZ$\\
        $d = ax_0 + by_0, x_0, y_0 \in \mathZ$\\
        $r = c - dq = a(x_1 -x_0q) + b(y_1 - y_0q) \in I$\\
        Но $r < d \overset{defn(d)}{\implies} r = 0 \implies c \in d\mathZ$
        \item $a = a1 + b0 \in I = d\mathZ \implies d \mid a$\\
        $b = a0 + b1 \in I = d\mathZ \implies d \mid b$\\
        $\letus d' \mid a,~d' \mid b,~d=ax_0 + by_0$\\
        $d' \mid ax_0,~d' \mid by_0 \implies d' \mid d$
    \end{enumerate}
\end{proof}

\begin{follow}~
    \begin{enumerate}
        \item $a, b \in \mathZ$: Тогда наибольший общий делитель $a, b$ существует.
        \item Если $d$ --- наибольший общий делитель $a, b$, то $\exists x, y \in \mathZ:~d = ax + by$(Линейное представление наибольшего общего делителя).
    \end{enumerate}
\end{follow}

\begin{proof}
    \begin{enumerate}
        \item Доказали в двух частях предложения.
        \item $\exists d_0$ --- наибольший общий делитель $a, b$, то есть $d_0 = ax_0 + by_0$\\
        $d$ ассоциирован с $d_0 \implies d = d_0\mathZ, z \in \mathZ \implies d = a(x_0z) + b(y_0z)$
    \end{enumerate}
\end{proof}

\begin{defn}
    НОД$(a, b) = \gcd(a, b)$ --- неотрицательный наибольший общий делитель $a, b$.
\end{defn}

\begin{theorem-non}
    $\letus a_1, a_2, b \in \mathZ: a_1 \equivm{b} a_2$\\
    Тогда $\gcd(a_1, b)$ = $\gcd(a_2, b)$.
\end{theorem-non}

\begin{proof}
    $(!)~\{c: c \mid a_1, c \mid b \} = \{c : c \mid a_2, c \mid b\}$\\ 
    "$\subset$":\\
    $a_2 - a_1 = bm \implies a_2 = a_1 + bm$\\
    $c \mid a_1, c \mid b \implies c \mid a_2$\\
    "$\supset$":\\
    $a_1 - a_2 = bm \implies a_1 = a_2 + bm$\\
    $c \mid a_2, c \mid b \implies c \mid a_1$\\
    $\forall x \in \{c: c \mid a_1, c \mid b \}: x \mid \gcd(a_1, b)$\\
    $\forall x \in \{c: c \mid a_2, c \mid b \}: x \mid \gcd(a_2, b)$\\
    $\gcd(a_1, b) = \gcd(a_2, b)$
\end{proof}

\begin{defn} 
    Алгоритм Евклида\\
    $\gcd(a, b) = \gcd(b, a \mod b)$, если $b \neq 0$\\
\end{defn}

\end{normalsize}