\begin{normalsize}

\section{Построение поля комплексных чисел}

\begin{defn}
    $\mathC = \mathR \times \mathR = \{(a, b) \mid a, b \in \mathR\}$\\
\end{defn}

\begin{defn} \
    \begin{itemize}
        \item Сложение на $\mathC$: $(a_1, b_1) + (a_2, b_2) = (a_1 + a_2, b_1 + b_2)$
        \item Умножение на $\mathC$: $(a_1, b_1) \cdot (a_2, b_2) = (a_1a_2 - b_1b_2, a_1b_2 + a_2b_1)$
    \end{itemize}
\end{defn}

\begin{theorem-non}
    $(\mathC, +, \cdot)$ - поле.
\end{theorem-non}

\begin{proof}
    \begin{itemize}
        \item Коммутативность сложения --- очевидно.
        \item Ассоциативность сложения --- очевидно.
        \item $(0, 0)$ --- нейтральный элемент сложения.
        \item $(-a, -b)$ --- обратный элемент к $(a, b)$.
        \item Коммутативность умножения --- очевидно.
        \item Ассоциативность умножения --- проверяется.
        \item Дистрибутивность --- проверяется.
        \item $(1, 0)$ --- нейтральный элемент умножения.
        \item $(a, b) z_1 z_2 = (1, 0):~z_1 = (a, -b),~z_2=\frac{1}{a^2+b^2}$
    \end{itemize}
\end{proof}

\begin{defn}
    $\mathC$ --- поле комплексных чисел.
\end{defn}

\begin{defn}
    $c \in \mathC$ --- комплексное число.
\end{defn}

\begin{theorem-non}
    $\mathR' = \{(a, 0) \mid a \in \mathR\}$\\
    $R'$ замкнуто относительно сложения, вычитания, умножения, содержит единицу, то есть является подкольцом поля $\mathC$.\\
    $\implies \mathR'$ --- само является кольцом относительно сложения, умножения, ограниченных на $\mathR'$. \\
    $\mathR \overset{\varphi}{\to} \mathR'(a \mapsto (a, 0))$, $\varphi(a)$ --- изоморфизм колец, т.е. $\varphi$ --- биекция и $\varphi(a + b) = \varphi(a) + \varphi(b)$; $\varphi(ab) = \varphi(a)\varphi(b)$.\\
    Отождествим $(a, 0)$ с вещественным числом $a$.\\
    $(a, 0) \cdot (0, 1) = (0, a)$\\
    $(a, b) = (a, 0) + (0, b) = (a, 0) + (b, 0) \cdot (0, 1) = a + b \cdot (0, 1) = a + bi$\\
\end{theorem-non}

\begin{defn}
    $z = a + bi$ --- комплексное число.
    $a = Re(z)$, $b = Im(z)$ --- действительная и мнимая части комплексного числа $z$.
    В геометрическом виде это вектор $z = (a, b)$.
\end{defn}

\begin{defn}
    $z = a + bi$ --- комплексное число. 
    $\overline{z} = a - bi$ --- сопряженное к $z$.
\end{defn}

\begin{defn}
    Автоморфизм --- изоморфизм на себя.
\end{defn}

\subsection*{Отступление про отображения}

\begin{defn}
    $id_M: M \to M,~x \mapsto x$ --- тождественное отображение на $M$.
\end{defn}

\begin{defn}
    $\letus \alpha: M \to N,~\beta: N \to P$ --- отображения\\
    Тогда $\alpha \circ \beta: M \to P,~x \mapsto \alpha(\beta(x))$ --- композиция отображений.
\end{defn}

\begin{defn}
    $\letus \alpha: M \to N$ --- отображение\\
    Отображение $\beta: N \to M$ --- обратное к $\alpha$, если $\beta \circ \alpha = id_M$.
\end{defn}

\begin{theorem-non}
    У отображения $\alpha: M \to N$ есть обратное отображение, если и только если $\alpha$ --- биекция.
\end{theorem-non}


\begin{proof}
    "$\Rightarrow$":\\
    Инъективность:\\
    $\beta \circ \alpha = id_M,~\alpha(x) = \alpha(y) \implies \beta(\alpha(x)) = \beta(\alpha(y)) \implies x = y$\\
    Сюръективность:\\
    $y \in N,~y = \alpha(\beta(y)) \in Im(\alpha)$(Im это прообраз)\\
    "$\Leftarrow$":\\
    Пусть $\alpha$ --- биекция, назовем $\beta: N \to M$ --- обратныи, если $\forall y \in N \alpha^{-1}(y) = \{x\},~x \in M$\\
    Положим $\beta(y) = x,~\alpha \circ \beta = id_N,~\beta \circ \alpha = id_M$
\end{proof}

\subsection*{Продолжение}

\begin{theorem-non}
    $\sigma: \mathC \to \mathC,~z \mapsto \overline{z}$ --- автоморфизм.
\end{theorem-non}

\begin{proof}
    $\sigma$ --- биекция, т.к. $\sigma \circ \sigma = id_{\mathC}$\\
    $\sigma(z_1 + z_2) = \sigma(z_1) + \sigma(z_2)$ --- очевидно\\
    $\sigma(z_1 z_2) = \sigma(z_1) \sigma(z_2)$\\
    $\sigma(1) = 1$ --- очевидно\\
    $z_1 = a_1 + b_1i,~z_2 = a_2 + b_2i$\\
    $\sigma(z_1 z_2) = \overline{a_1 a_2 - b_1 b_2 + i(a_1 b_2 + a_2 b_1)} = a_1 a_2 - b_1 b_2 + i(a_1 b_2 + a_2 b_1)$\\
    $\sigma(z_1) \sigma(z_2) = \overline{(a_1 - i b_1) (a_2 - i b_2)} = a_1 a_2 - b_1 b_2 + i(a_1 b_2 + a_2 b_1)$\\
\end{proof}

\end{normalsize}