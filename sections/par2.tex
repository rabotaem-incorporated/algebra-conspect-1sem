\subsection{Отношение эквивалентности и разбиение на классы}

\begin{defn}
    Отношение эквивалентности --- бинарное отношение, удовлетворяющее следующим свойствам: рефлексивность, симметричность, транзитивность.
\end{defn}

\begin{defn}
    Разбиение на классы множества $M$ --- это представление $M$ в виде $M = \bigcup\limits_{i \in I} M_i$, где $M_i$ --- классы, $I$ --- индексное множество, $M_i \cap M_j = \emptyset$ при $i \neq j$.
\end{defn}

\begin{theorem}
    Пусть $M = \bigcup\limits_{i \in I} M_i$ --- разбиение на классы, тогда $a \sim b \iff \exists i : a, b \in M_i$.
\end{theorem}

\begin{proof}~

    рефлексивность, симметричность --- очевидны
    
    транзитивность: $a \sim b, b \sim c \implies \exists i, j : a, b \in M_i$ и $b, c \in M_j$
    
    $b \in M_i \cap M_j \iff M_i \cap M_j \neq \emptyset \implies i = j \implies a, c \in M_i \implies a \sim c$
\end{proof}

\begin{theorem}
    $\letus \sim$ --- отношение эквивалентности на $M$. Значит $\exists$ разбиение на классы $M = \bigcup\limits_{i \in I} M_i$ такое, что $\forall a, b \in M: a \sim b \iff \exists i : a, b \in M_i$.
\end{theorem}

\begin{proof}~
    
    $[a] = \{b \in M \mid a \sim b\}$ --- класс, $a \in M$
    
    $\forall a_1, a_2 \in M : [a_1] \cap [a_2] = \emptyset$ или $[a_1] = [a_2]$
    
    $\letus [a_1] \cap [a_2] \neq \emptyset \implies \exists x \in [a_1] \cap [a_2]$
    
    $x \in [a_1], x \in [a_2] \implies x \sim a_1, x \sim a_2 \implies a_2 \sim a_1$
    
    $[a_2] \subset [a_1], c \in [a_2] \implies c \sim a_2 \implies c \sim a_1 \implies c \in [a_1]$
    
    $[a_1] \subset [a_2], c \in [a_1] \implies c \sim a_1 \implies c \sim a_2 \implies c \in [a_2]$
    
    Значит $[a_1] = [a_2]$
    
    $I = \{[a] \mid a \in M\}$
    
    $\forall \mathfrak{A}, \mathfrak{B} \in I: \mathfrak{A} \cap \mathfrak{B} = \emptyset$
    
    $a_1, a_2 \in \mathfrak{A} \implies [a_1] = \mathfrak{A} = [a_2] \implies a_2 \in [a_1] \implies a_2 \sim a_1$\
    
    $a_1 \in \mathfrak{A}, a_2 \in \mathfrak{B} \implies \neg(a_1 \sim a_2)$, так как иначе $a_1 \in [a_2] \implies \mathfrak{B} \in \mathfrak{A} \implies \mathfrak{A} \cap \mathfrak{B} \neq \emptyset$
\end{proof}

\begin{defn}
    Фактор-множество по отношению эквивалентности $\sim$ --- множество $I$, обозначим его как $M/\sim$
\end{defn}

\begin{example}
    $\Z/\sim = \{[z] \mid z \in \Z \} = \{[0], [1], [2], \dots\}$
\end{example}