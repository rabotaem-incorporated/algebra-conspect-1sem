\subsection{Кольцо классов вычетов}

\begin{defn}
    Множество классов вычетов по модулю $m$ --- это множество всех вычетов по модулю $m$.

    Обозначается как $\Z/m\Z \iff \Z/m \iff \Z/\equivm{m}$
\end{defn}

\begin{theorem}

    $\letus m \in \N$. Тогда 
    
    \begin{enumerate}
        \item $\Z/m\Z = \{\oln{0}, \oln{1}, \ldots, \oln{m-1}\}$
        \item $|\Z/m\Z| = m$ 
    \end{enumerate}
\end{theorem}

\begin{proof}~

    \begin{enumerate}
        \item $a \in \Z$ $(!) \oln{a} = \oln{r},~0 \leq r < m$
        \begin{enumerate}
            \item[a)] $a \geq 0$, $\letus r$ --- наименьшее число, такое что $r \geq 0, a \equivm{m} r$
            
            $r \geq m \implies r - m \equivm{m} a, r - m \geq 0, r - m < r$. Противоречие с выбором $r$.
            
            Значит $r < m$, тоесть $r$ --- искомое.
            
            \item[b)] $a < 0,~a' = a \pm (-a)m = a(1-m) \geq 0$
            
            $\oln{a} = \oln{a'} = \oln{r},~0 \leq r < m$
        \end{enumerate}
        
        \item предположим $\oln{r} = \oln{r'},~0 \leq r, r' < m$.
        
        $|r' - r| < m \implies m \mid (r - r') \implies r' - r = 0$
    \end{enumerate}
\end{proof}

\begin{follow}~

    Теорема о делениии с остатком --- $\letus a \in \Z, b \in \N \implies \exists! q, r \in \Z$
    \begin{enumerate}
        \item $a = bq + r,~0 \leq r < b$
        \item $0 \leq r < b$
    \end{enumerate}
\end{follow}

\begin{proof}~

    \textsl{Существование:}

    В $\Z/b\Z$ рассмотрим $\oln{a} \in \{\oln{0}, \oln{1}, \ldots, \oln{b-1}\}$, тогда если $\oln{a} = \oln{r},~0 \leq r < b$

    $a \equivm{b} r \iff a = bq + r,~q \in \Z$

    \textsl{Единственность:}

    $\letus a = bq + r = bq' + r',~0 \leq r, r' < b \iff \oln{bq+r} = \oln{bq'+r'} \iff \oln{r} = \oln{r'} \iff r = r' \implies bq = bq' \implies q = q'$

\end{proof}

\begin{defn}
    $q$ --- неполное частное при делении $a$ на $b$, $r$ --- остаток при делении $a$ на $b$
\end{defn}

\begin{defn}
    Операция на множестве $M$ --- бинарная операция $M \times M \to M$
\end{defn}

На $\Z/m\Z$ определим операцию сложения и умножения по модулю $m$:

\begin{itemize}
    \item $\oln{a} + \oln{b} = \oln{a + b}$
    
    \item $\oln{a} \cdot \oln{b} = \oln{a \cdot b}$
\end{itemize}

$(!)~\oln{a} = \oln{a'},~\oln{b} = \oln{b'} \implies \oln{a + b} = \oln{a' + b'},~\oln{a \cdot b} = \oln{a' \cdot b'}$

$\oln{a} = \oln{a'},~\oln{b} = \oln{b'} \implies a \equivm{m} a',~\implies b \equivm{m} b' \implies a + b \equivm{m} a' + b',~ a \cdot b \equivm{m} a' \cdot b' \implies$

$\oln{a + b} = \oln{a' + b'},~\oln{a \cdot b} = \oln{a' \cdot b'}$

\begin{example}
    $m = 4, \Z/4\Z = \{\oln{0}, \oln{1}, \oln{2}, \oln{3}\}$
\end{example}

\[
\begin{tabular}{c|cccc}
    $+$ & $\oln{0}$ & $\oln{1}$ & $\oln{2}$ & $\oln{3}$\\
    \hline
    $\oln{0}$ & $\oln{0}$ & $\oln{1}$ & $\oln{2}$ & $\oln{3}$\\
    $\oln{1}$ & $\oln{1}$ & $\oln{2}$ & $\oln{3}$ & $\oln{0}$\\
    $\oln{2}$ & $\oln{2}$ & $\oln{3}$ & $\oln{0}$ & $\oln{1}$\\
    $\oln{3}$ & $\oln{3}$ & $\oln{0}$ & $\oln{1}$ & $\oln{2}$
\end{tabular}
\quad\quad\quad
\begin{tabular}{c|cccc}
    $\cdot$ & $\oln{0}$ & $\oln{1}$ & $\oln{2}$ & $\oln{3}$\\
    \hline
    $\oln{0}$ & $\oln{0}$ & $\oln{0}$ & $\oln{0}$ & $\oln{0}$\\
    $\oln{1}$ & $\oln{0}$ & $\oln{1}$ & $\oln{2}$ & $\oln{3}$\\
    $\oln{2}$ & $\oln{0}$ & $\oln{2}$ & $\oln{0}$ & $\oln{2}$\\
    $\oln{3}$ & $\oln{0}$ & $\oln{3}$ & $\oln{2}$ & $\oln{1}$
\end{tabular}
\]

\begin{defn}
    $e \in M$ --- нейтральный элемент относительно операции(*) на $M$, если $\forall a \in M$ справедливо $a * e = e * a = a$
\end{defn}

\begin{theorem-non}
    Операции сложения и умножения на $\Z/m\Z$ обладают следующими свойствами:
\end{theorem-non}

\begin{enumerate}
    \item $A + B = B + A$ --- коммутативность сложения
    
    \item $(A + B) + C = A + (B + C)$ --- ассоциативность сложения
    
    \item $A + \oln{0} = A$--- существование нейтрального элемента относительно сложения
    
    \item $A + A' = \oln{0}$ --- существование обратного элемента относительно сложения
    
    \item $AB = BA$ --- коммутативность умножения
    
    \item $(AB)C = A(BC)$ --- ассоциативность умножения
    
    \item $A \cdot \oln{1} = A$ --- существование нейтрального элемента относительно умножения
    
    \item $A \cdot (B + C) = A \cdot B + A \cdot C$ --- дистрибутивность умножения относительно сложения.
    
    \item $(B + C) \cdot A = B \cdot A + C \cdot A$ --- дистрибутивность сложения относительно умножения.
\end{enumerate}

\begin{defn}
    Кольцом называется множество $M$ с операциями сложения и умножения, для которых выполнены аналоги свойств 1-4 и 8-9.
\end{defn}

\begin{defn}
    Кольцо коммутативное, если выполнены свойство 5.
\end{defn}

\begin{defn}
    Колько ассоциативное, если выполнено свойство 6.
\end{defn}

\begin{defn}
    Кольцо c единицей, если выполнено свойство 7.
\end{defn}

\begin{defn}
    $\forall x \in \R \exists y \in \R: x + y = n \implies n$ --- нейтральный элемент относительно сложения.
\end{defn}    

\begin{notice}
    Если (*) --- операция на $M$, то существует единственный нейтральный элемент относительно (*).
\end{notice}

\begin{proof}
    $e, e'$ --- нейтральные элементы относительно (*), тогда $e = e * e' = e'$.
\end{proof}

\begin{theorem-non}
    {\textbf{В нашем курсе все кольца будут ассоциативные с единицей.}}
\end{theorem-non}

\begin{lemma}
    В любом кольце $0 \cdot a = 0$.
\end{lemma}
\begin{proof}

    $0 + 0 = 0 \implies (0 + 0) \cdot a = 0 \cdot a \implies 0 \cdot a + 0 \cdot a = 0 \cdot a$

    $\letus 0 \cdot A \neq 0 \implies \exists b: b + 0 \cdot A = 0$
    
    $0 = b + 0 \cdot a = b + (0 \cdot a + 0 \cdot a) = (b + 0 \cdot a) + (0 \cdot a) = 0 + (0 \cdot a) = (0 \cdot a)$
\end{proof}

\begin{defn}
    $A^*$ --- множество обратимых элементов $A$.
\end{defn}

\begin{examples}
    \begin{itemize}
        \item $\R^* = \R \setminus \{0\}$
        \item $\Z^* = \{-1, 1\}$
        \item $(\Z/4\Z)^* = \{\oln{1}, \oln{3}\}$
        \item $(\Z/5\Z)^* = \{\oln{1}, \oln{2}, \oln{3}, \oln{4}\}$
    \end{itemize}
\end{examples}

\begin{defn}
    Полем называется коммутативное кольцо $F$, такое что $F^* = F \setminus \{0\}$.
\end{defn}