\subsection{Кольцо классов вычетов}

\begin{defn}
    Множество классов вычетов по модулю $m$ --- это множество всех вычетов по модулю $m$.

    Обозначается как $\Z/m\Z \iff \Z/m \iff \Z/\equivm{m}$
\end{defn}

\begin{theorem}

    $\letus m \in \N$. Тогда 
    
    \begin{enumerate}
        \item $\Z/m\Z = \{\oln{0}, \oln{1}, \ldots, \oln{m-1}\}$
        \item $|\Z/m\Z| = m$ 
    \end{enumerate}
\end{theorem}

\begin{proof}

    \begin{enumerate}
        \item $\letus a \in \Z$, $\prove \oln{a} = \oln{r}, \quad 0 \leq r < m$
        \begin{enumerate}
            \item[a)] Случай $a \geq 0$: $\letus r$ --- наименьшее число, такое что $r \geq 0$ и $a \equivm{m} r$.
            
            Если $r \geq m$, то $r - m \equivm{m} a,~ r - m \geq 0,~ r - m < r$. 
            То есть $r - m$ подходит под условие для $r$ и меньше. 
            Противоречие с выбором $r$.
            
            Значит $r < m$, то есть $r$ --- искомое.
            
            \item[b)] Случай $a < 0$: 
            
            Рассмотрим $a' = a \pm (-a)m = a(1-m)$. Тогда $a < 0,~1 - m \le 0$, и $a' \ge 0$.
            
            $\oln{a} = \oln{a'} = \oln{r},~ 0 \leq r < m$
        \end{enumerate}
        
        \item предположим $\oln{r} = \oln{r'},~ 0 \leq r, r' < m$.
        \begin{gather*}
            \begin{cases}|r' - r| < m \\ m \mid (r - r')\end{cases} \implies r' - r = 0 \implies r = r'.
        \end{gather*}
    \end{enumerate}
\end{proof}

\begin{follow}
    Теорема о делении с остатком
    
    Пусть $a \in \Z,~b \in \N$. Тогда 
    \begin{gather*}
        \exists!\,q, r \in \Z : \begin{cases}a = bq + r \\ \leq r < b\end{cases}
    \end{gather*}
\end{follow}

\begin{proof}

    <<Существование>>:

    В $\Z/b\Z$ рассмотрим $\oln{a} \in \{\oln{0}, \oln{1}, \ldots, \oln{b-1}\}$, тогда по теореме выше найдется $0 \leq r < b$ для которого $\oln{a} = \oln{r}$:
    \begin{gather*}
        a \equivm{b} r \iff a = bq + r, \quad q \in \Z.
    \end{gather*}

    <<Единственность>>:
    Пусть нашлось два таких $q, q' \in \Z$ и $r, r' \in \Z$ для которых $a = bq + r,~ a = bq' + r'$. Тогда
    \begin{gather*}
        bq+r \equivm{b} bq'+r' \iff r \equivm{b} r' \lbl{$0 \leq r, r' < b$}\iff r = r' \implies bq = bq' \iff q = q'.
    \end{gather*}
    Напомню, что вторая равносильнось выполняется благодаря единственности класса вычетов $\oln{r}$.
\end{proof}

\begin{defn}
    $q$ --- \emph{неполное частное} при делении $a$ на $b$, $r$ --- \emph{остаток} при делении $a$ на $b$.
\end{defn}

\begin{defn}
    \emph{Операция} на множестве $M$ --- бинарное отображение $M \times M \to M$.
\end{defn}

На $\Z/m\Z$ определим операцию сложения и умножения по модулю $m$:

\begin{itemize}
    \item $\oln{a} + \oln{b} = \oln{a + b}$
    
    \item $\oln{a} \cdot \oln{b} = \oln{a \cdot b}$
\end{itemize}

\begin{theorem-non}
    Это правда операции над множеством $\Z/m\Z$:
\end{theorem-non}
\begin{proof}
    То, что за пределы множества при сложении и умножении мы не выходим, очевидно.
    Надо доказать, что при подстановке одинаковых классов, получаеются одинаковые результаты, то есть:
    \begin{gather*}
        \prove \oln{a} = \oln{a'},~ \oln{b} = \oln{b'} \implies \oln{a + b} = \oln{a' + b'},~\oln{a \cdot b} = \oln{a' \cdot b'}
    \end{gather*}
    распишем условия через сравнения по модулю:
    \begin{gather*}
        \oln{a} = \oln{a'},~\oln{b} = \oln{b'} \implies a \equivm{m} a',~ b \equivm{m} b'
    \end{gather*}
    Воспользуемся свойствами сравнения:
    \begin{gather*}
        a \equivm{m} a',~ b \equivm{m} b' \implies a + b \equivm{m} a' + b',~ a \cdot b \equivm{m} a' \cdot b'
    \end{gather*}
    И перейдем обратно к классам:
    \begin{gather*}
        a + b \equivm{m} a' + b',~ a \cdot b \equivm{m} a' \cdot b' \implies \oln{a + b} = \oln{a' + b'}, \quad \oln{a \cdot b} = \oln{a' \cdot b'}
    \end{gather*}
\end{proof}

\begin{example}
    $m = 4,~ \Z/4\Z = \{\oln{0}, \oln{1}, \oln{2}, \oln{3}\}$
\end{example}

\[
\begin{tabular}{c|cccc}
    $+$ & $\oln{0}$ & $\oln{1}$ & $\oln{2}$ & $\oln{3}$\\
    \hline
    $\oln{0}$ & $\oln{0}$ & $\oln{1}$ & $\oln{2}$ & $\oln{3}$\\
    $\oln{1}$ & $\oln{1}$ & $\oln{2}$ & $\oln{3}$ & $\oln{0}$\\
    $\oln{2}$ & $\oln{2}$ & $\oln{3}$ & $\oln{0}$ & $\oln{1}$\\
    $\oln{3}$ & $\oln{3}$ & $\oln{0}$ & $\oln{1}$ & $\oln{2}$
\end{tabular}
\quad\quad\quad
\begin{tabular}{c|cccc}
    $\cdot$ & $\oln{0}$ & $\oln{1}$ & $\oln{2}$ & $\oln{3}$\\
    \hline
    $\oln{0}$ & $\oln{0}$ & $\oln{0}$ & $\oln{0}$ & $\oln{0}$\\
    $\oln{1}$ & $\oln{0}$ & $\oln{1}$ & $\oln{2}$ & $\oln{3}$\\
    $\oln{2}$ & $\oln{0}$ & $\oln{2}$ & $\oln{0}$ & $\oln{2}$\\
    $\oln{3}$ & $\oln{0}$ & $\oln{3}$ & $\oln{2}$ & $\oln{1}$
\end{tabular}
\]

\begin{defn}
    $e \in M$ --- \emph{нейтральный элемент} относительно операции $*$ на $M$, если $\forall a \in M$ справедливо $a * e = e * a = a$.
\end{defn}

\begin{theorem-non}
    Операции сложения и умножения на $\Z/m\Z$ обладают следующими свойствами:
\end{theorem-non}

$\forall A,~B,~C \quad \exists A'$:
\begin{enumerate}
    \item $A + B = B + A$ --- коммутативность сложения
    
    \item $(A + B) + C = A + (B + C)$ --- ассоциативность сложения
    
    \item $A + \oln{0} = A$--- существование нейтрального элемента относительно сложения
    
    \item $A + A' = \oln{0}$ --- существование обратного элемента относительно сложения
    
    \item $AB = BA$ --- коммутативность умножения
    
    \item $(AB)C = A(BC)$ --- ассоциативность умножения
    
    \item $A \cdot \oln{1} = A$ --- существование нейтрального элемента относительно умножения
    
    \item $A \cdot (B + C) = A \cdot B + A \cdot C$ --- дистрибутивность умножения относительно сложения.
    
    \item $(B + C) \cdot A = B \cdot A + C \cdot A$ --- дистрибутивность сложения относительно умножения.
\end{enumerate}

\begin{defn}
    \emph{Кольцом} называется множество $M$ с операциями сложения и умножения, для которых выполнены аналоги свойств 1-4 и 8-9.
\end{defn}

\begin{defn}
    Кольцо \emph{коммутативное}, если выполнено свойство 5.
\end{defn}

\begin{defn}
    Колько \emph{ассоциативное}, если выполнено свойство 6.
\end{defn}

\begin{defn}
    Кольцо \emph{c единицей}, если выполнено свойство 7.
\end{defn}

\begin{defn}
    Я оставлю это для полноты картины, но wtf is this?

    % выглядит как какой-то факт, весьма очевидный, но факт (ну либо это было пояснение к определению)
    \sout{$\forall x \in \R \quad \exists y \in \R: x + y = n \implies n$ --- нейтральный элемент относительно сложения.}
\end{defn}

\begin{notice}
    Если $*$ --- операция на $M$, то существует единственный нейтральный элемент относительно $*$.
\end{notice}

\begin{proof}
    $e,~ e'$ --- нейтральные элементы относительно $*$, тогда $e = e * e' = e'$.
    
    Типа просто в определение нейтрального элемента подставили и получилось.
\end{proof}

\begin{theorem-non}
    {\textbf{В нашем курсе все кольца будут ассоциативные с единицей.}}
\end{theorem-non}

\begin{lemma}
    В любом кольце $0 \cdot a = 0$.
\end{lemma}
\begin{proof}~

    Предположим противное.
    Покажем, что $0 \cdot a + 0 \cdot a = 0 \cdot a$.
    \begin{gather*}
        0 + 0 = 0 \lbl{$\exists 0$}\implies (0 + 0) \cdot a = 0 \cdot a \lbl{дистр.}\implies 0 \cdot a + 0 \cdot a = 0 \cdot a
    \end{gather*}
    Теперь вычтем $0 \cdot a$. Так как $\exists b:\ b + (0 \cdot a) = 0$, то
    \begin{gather*}
        0 = b + (0 \cdot a) = b + (0 \cdot a + 0 \cdot a) = (b + 0 \cdot a) + (0 \cdot a) = 0 + (0 \cdot a) = 0 \cdot a
    \end{gather*}
    Противоречие.
\end{proof}

\begin{defn}
    $A^*$ --- множество обратимых элементов кольца $A$ (по умножению, разумеется).
\end{defn}

\begin{examples}~
    \begin{itemize}
        \item $\R^* = \R \setminus \{0\}$
        \item $\Z^* = \{-1, 1\}$
        \item $(\Z/4\Z)^* = \{\oln{1}, \oln{3}\}$
        \item $(\Z/5\Z)^* = \{\oln{1}, \oln{2}, \oln{3}, \oln{4}\}$
    \end{itemize}
\end{examples}

\begin{defn}
    \emph{Полем} называется коммутативное кольцо $F$, такое что $F^* = F \setminus \{0\}$.
\end{defn}
