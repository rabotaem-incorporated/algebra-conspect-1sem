\subsection{Наибольший общий делитель}

\begin{defn}
    $R$ --- коммутативное кольцо, $a, b \in R$.

    Элемент $d$ называется наибольшим общим делителем, если:

    \begin{enumerate}
        \item $d \mid a,~ d \mid b$
        \item $d' \mid a,~ d' \mid b \implies d' \mid d$
    \end{enumerate}
\end{defn}

\begin{theorem-non}~
    \begin{enumerate}
        \item $d_1,~ d_2$ --- наибольшие общие делители, тогда $d_1 \sim d_2$.
        
        \item $\letus d_1$ --- наибольший общий делитель, $d_2 \sim d_1$, тогда $d_2$ --- тоже наибольший общий делитель.
    \end{enumerate}
\end{theorem-non}

\begin{proof}
    \begin{enumerate}
        \item По свойству 2 : $d_1 \mid d_2,~d_2 \mid d_1 \implies d_1 \sim d_2$.
        
        \item $d_2 \mid d_1,~d_1 \mid a,~d_1 \mid b \implies d_2 \mid a,~d_2 \mid b$
        
        Пусть $d_2$ не наибольший, тогда $\exists d' > d_2$.
        
        $d' \mid a,~d' \mid b \implies d' \mid d_1$, так как $d_1$ наибольший общий делитель,
        
        $d' \mid d_1,~d_1 \mid d_2 \implies d' \mid d_2$, противоречие, так как $d' > d_2$.
    \end{enumerate}
\end{proof}

\begin{theorem-non} 
    $\letus a, b \in \Z \implies$
    
    \begin{enumerate}
        \item $\exists d \in \Z:~a\Z + b\Z = d\Z$, иначе говоря: $\forall x, y \in \Z \ \exists d, z \in \Z: ax + by = dz$
        
        \item при этом $d$ --- наибольший общий делитель $a, b$.
    \end{enumerate}
\end{theorem-non}

\begin{proof}~
    \begin{enumerate}
        \item Пусть $I = a\Z + b\Z$.
        
        Заметим что $0 \in I$, так как $0a + 0b = 0$.
        
        Если $I = \{0\}$, то $I = 0\Z$.
        
        Иначе $I \neq \{0\} \implies c \in I \implies -c \in I$, так как $-(ax + by) = a \cdot -x + b \cdot -y$
        
        То есть в $I$ есть положительные числа.
        
        Пусть $d = \min\{ c \mid c \in I,~ c > 0 \}$, и докажем что $a\Z + b\Z = d\Z$.
        
        <<$\supset$>>:
        
        $d \in I$ (по определению) $\implies d = ax_0 + by_0, \quad x_0, y_0 \in \Z \implies$
        
        $\forall z \in \Z:\ dz = a(x_0z) + b(y_0z) \in I$, значит $d\Z \subset a\Z + b\Z$
        
        <<$\subset$>>:
        
        $\letus c \in I,~ d \in \N \implies \exists q, r \in \Z:\ c = dq + r, \quad 0 \leq r < d$
        
        $c \in I$, значит $c = ax_1 + by_1, \quad x_1, y_1 \in \Z$
        
        Мы уже знаем, что $d \in I$, значит $d = ax_0 + by_0, \quad x_0, y_0 \in \Z$ 
        
        $r = c - dq = a(x_1 -x_0q) + b(y_1 - y_0q) \in I$

        По определению остатка: $\begin{cases} 
            r \geq 0,\\
            r < d 
        \end{cases}$, но $d = \min\{ c \mid c \in I,~ c > 0 \} \implies$ 
        
        $\implies c \in d\Z \implies a\Z + b\Z \subset d\Z$
        
        \item Пусть $d$ --- наибольший общий делитель $a, b$.
        
        $a = a1 + b0 \in I = d\Z \implies d \mid a$
        
        $b = a0 + b1 \in I = d\Z \implies d \mid b$
        
        $\letus d' \mid a,~d' \mid b,~d=ax_0 + by_0$
        
        $d' \mid ax_0,~d' \mid by_0 \implies d' \mid d$, значит $d$ действительно наибольший общий делитель $a, b$.
    \end{enumerate}
\end{proof}

\begin{follow}~
    \begin{enumerate}
        \item $a, b \in \Z:$ Тогда наибольший общий делитель $a, b$ существует.
        
        \item Если $d$ --- наибольший общий делитель $a, b$, то $\exists x, y \in \Z:~d = ax + by$ \quad (\emph{Линейное представление наибольшего общего делителя}).
    \end{enumerate}
\end{follow}

\begin{proof}
    \begin{enumerate}
        \item Доказали в двух частях предложения.
        
        \item Из первой части знаем, что существует $d_0$ --- наибольший общий делитель $a, b$, то есть $d_0 = ax_0 + by_0$
        
        $d$ ассоциирован с $d_0 \implies d = d_0\Z,~ z \in \Z \implies d = a(x_0z) + b(y_0z)$
    \end{enumerate}
\end{proof}

\begin{defn}
    НОД$(a, b) \iff \gcd(a, b)$ --- неотрицательный наибольший общий делитель $a, b$.
\end{defn}

\begin{theorem-non}
    $\letus a_1, a_2, b \in \Z:\ a_1 \equivm{b} a_2$
    
    Тогда $\gcd(a_1, b)$ = $\gcd(a_2, b)$.
\end{theorem-non}

\begin{proof}
    $\prove \{c: c \mid a_1,~ c \mid b \} = \{c : c \mid a_2,~ c \mid b\}$
    
    <<$\subset$>>:
    
    $a_2 - a_1 = bm \implies a_2 = a_1 + bm$
    
    $c \mid a_1,~c \mid b \implies c \mid a_2$

    <<$\supset$>>:
    
    $a_1 - a_2 = bm \implies a_1 = a_2 + bm$

    $c \mid a_2,~ c \mid b \implies c \mid a_1$

    Получается, что:
    
    $\forall x \in \{c: c \mid a_1, c \mid b \}:\ x \mid \gcd(a_1, b)$
    
    $\forall x \in \{c: c \mid a_2, c \mid b \}:\ x \mid \gcd(a_2, b)$
    
    $\gcd(a_1, b) = \gcd(a_2, b)$
\end{proof}

\begin{defn} 
    Алгоритм Евклида

    $\gcd(a, b) = \gcd(b, a \mod b)$, если $b \neq 0$
\end{defn}

\lstinputlisting[style=supercpp]{code/gcd.cpp}