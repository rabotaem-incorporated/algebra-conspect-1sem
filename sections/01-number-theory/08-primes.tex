\subsection{Простые числа}

\begin{defn}
    Число $p \in \Z$ называется простым, если $p \notin \{-1, 0, 1\}$ и все делители $p$ --- это $\pm 1$ и $p$.
\end{defn}

\begin{prop}~
    \begin{enumerate}
        \item $p$ --- простое $\iff -p$ --- простое.
        
        \item $p$ --- простое, $a \in \Z \implies p \mid a$ или $p \bot a$.
        
        \item $p, q$ --- простые $\implies p \sim q$ или $p \bot q$.
        
        \item $p \mid ab \implies p \mid a$ или $p \mid b$. 
    \end{enumerate}
\end{prop}

\begin{theorem-non}
    $\letus a \neq \pm 1$, тогда существует простое число $p :\ p \mid a$.
\end{theorem-non}

\begin{proof}~

    Пусть $a = 0$, тогда $p = 239$
    
    Тогда $a \neq 0$, пускай $a > 0$, так как, случай $a < 0$ аналогичен.
    
    Индукция по $a$:

    <<База>>: $a = 1$, но $a > 0$, значит простое число уже встречалось.
    
    <<Переход>>:
    
    $a$ --- простое $\implies p = a,~ p \mid a$
    
    $a$ --- не простое, значит $\exists d :\quad 1 < d < a,~ d \mid a$
    
    $a = dd'$, тогда по индукционному переходу существует простое число $p :\ p \mid d$
    
    $p \mid d,~d \mid a \implies p \mid a$
\end{proof}

\begin{defn}
    Составное число --- это число отличное от 0, и не являющееся простым.
\end{defn}

\begin{defn}
    Решето Эратосфена --- это алгоритм, который позволяет найти все простые числа от 1 до $n$.

    $2, 3, 4, 5, 6, 7, 8, 9, \ldots, 100$

    \begin{itemize}
        \item 2 --- простое, вычеркиваем все числа кратные 2
        
        \item 3 --- простое, вычеркиваем все числа кратные 3
        
        \item 4 --- составное, пропускаем
        
        \item и т. д.
    \end{itemize}

    В итоге получим все простые числа от $1$ до $100$.
\end{defn}

\begin{notice}
    $\P$ --- множество всех простых чисел.
\end{notice}

\begin{theorem}[Теорема Евклида]
    Существует бесконечно много простых чисел
\end{theorem}

\begin{proof}

    $\letus p_1, p_2, \ldots p_n$ --- все простые числа. Возьмем $N = p_1 p_2 \ldots p_n + 1$, пусть оно составное $\implies$
    
    $\exists p \in \P :\ p \mid N,~ p > 0 \implies \exists j:\ p = p_j$
    
    Тогда, $p \mid (N - 1) \implies p \mid 1 \implies p = \pm 1$, противоречие.    
\end{proof}