\subsection{Китайская теорема об остатках}

\begin{theorem}
    Пусть $m_1 \bot m_2,\quad a_1, a_2 \in \Z$, тогда:

    \begin{enumerate}
        \item $\exists x_0 \in \Z$: 
        $\begin{cases}
            x_0 \equivm{m_1} a_1 \\
            x_0 \equivm{m_2} a_2
        \end{cases}$
        
        \item $\letus x_0$ удовлетворяет системе выше, тогда:
        
        $x \in \Z$, где $x$ удовлетворяет системе выше $\iff x \equivm{m_1 m_2} x_0$
    \end{enumerate}
\end{theorem}

\begin{proof}
    \begin{enumerate}
        \item $x_0 = a_1 + km_1 = a_2 + lm_2 \implies km_1 - lm_2 = a_2 - a_1$ --- линейное диофантово уравнение с двумя неизвестными $k, l$
        
        $m_1 \bot m_2 \implies$ у него есть решение $(k_0, l_0)$
        
        $x_0 = a_1 + k_0 m_1$ --- искомое

        \item
        <<$\Leftarrow$>>:
        $x \equivm{m_1m_2} x_0 \implies 
        \begin {cases} 
            x \equivm{m_1} x_0 \\
            x \equivm{m_2} x_0
        \end{cases} \implies
        \begin{cases}
            x \equivm{m_1} a_1 \\
            x \equivm{m_2} a_2
        \end{cases}$

        <<$\Rightarrow$>>:
        $x$ удовлетворяет системе из теоремы $\implies
        \begin {cases} 
            x \equivm{m_1} x_0 \\
            x \equivm{m_2} x_0
        \end{cases} \implies
        \begin{cases}
            m_1 \mid (x - x_0)\\
            m_2 \mid (x - x_0)
        \end{cases} \lbl{$m_1 \bot m_2$}\implies m_1m_2 \mid (x - x_0)$
    \end{enumerate}
\end{proof}

\begin{defn}
    $\letus R, S$ --- кольца с единицей. Отображение $\phi: R \to S$ называется изоморфизмом колец, если: $\phi$ биекция.
    
    \begin{enumerate}
        \item $\forall\ r_1,~ r_2:\ \phi(r_1 + r_2) = \phi(r_1) + \phi(r_2)$
        
        \item $\forall\ r_1,~ r_2:\ \phi(r_1 r_2) = \phi(r_1) \phi(r_2)$
    \end{enumerate}
\end{defn}

\begin{theorem-non}
    $\letus (m_1, m_2) = 1$ 

    Тогда существует изоморфизм 
    
    $\Z/m_1m_2\Z \to \Z/m_1\Z \times \Z/m_2\Z$
    
    $[a]_{m_1m_2} \mapsto ([a]_{m_1}, [a]_{m_2})$
\end{theorem-non}

\begin{proof}
    Проверим корректность:
    
    $\letus [a]_{m_1m_2} = [a']_{m_1m_2} \implies$
    $a \equivm{m_1m_2} a' \implies$
    $\begin{cases}
        a \equivm{m_1} a' \\
        a \equivm{m_2} a'
    \end{cases} \implies$
    $([a]_{m_1}, [a]_{m_2}) = ([a']_{m_1}, [a']_{m_2})$
    
    $\phi([a]_{m_1m_2} + [b]_{m_1m_2}) = \phi([a + b]_{m_1m_2}) = ([a + b]_{m_1}, [a + b]_{m_2}) = $
    
    $([a]_{m_1}, [a]_{m_2}) + ([b]_{m_1}, [b]_{m_2}) = \phi([a]_{m_1m_2}) + \phi([b]_{m_1m_2})$
    
    Для умножения аналогично.
    
    $\phi$ - отображение между конечными равномощными множествами, поэтому оно биективно $\iff$ оно сюръективно $\iff$ оно инъективно.

    Действительно, если $\phi: A \to B,~ |A| = |B| < \infty$ инъективно, 
    то полный прообраз любого элемента из $B$ состоит из не более чем одного элемента из $A$ (определение инъективности). 

    А если сложить количества прообразов у всех элементов из $B$, то должно получиться в точности $|A|$, так как каждый прообраз - чей-то образ. 

    Но тогда каждый прообраз состоит из в точности одного элемента, т. е. $\phi$ - биекция. 
    
    Аналогично можно рассуждать и про сюрьективное отображение.

    Проверим сюръективность $\phi$
    
    По китайской теореме об остатках $\forall a_1, a_2 \in \Z~ \exists a \in \Z: 
    \begin{cases}
        a \equivm{m_1} a_1 \\
        a \equivm{m_2} a_2 \\
    \end{cases}$

    Таким образом $\phi$ - биекция.
\end{proof}


