\subsection{Линейные диофантовы уравнения}

\begin{defn}
    \emph{Линейным диофантовым уравнением} с двумя неизвестными называется уравнение вида $ax + by = c$, где $a, b, c \in \Z$.
\end{defn}

\begin{defn}
    \emph{Решением линейного диофантова уравнения} называется множество всех пар $(x, y) \in \Z^2:\ ax + by = c$.
\end{defn}

\begin{notice}
    Если $\gcd(a, b) \nmid c$, то решение --- пустое множество, так как все линейные комбинации $a, b$ делятся на $\gcd(a, b)$.
\end{notice}

\begin{notice}
    Теперь заметим следующее: если $ax_1 + by_1 = c$ и $ax_2 + by_2 = c$, то $a(x_1 - x_2) + b(y_1 - y_2) = 0$.
    Иными словами, разность двух решений линейного диофантова уравнения --- решение соответствующего однородного уравнения.
\end{notice}

А значит все решения линейного диофантова уравнения можно найти, решив однородное уравнение и прибавив ко всем его решениям какое-то решение исходного уравнения.

Решим однородное уравнение:

$ax + by = 0 \iff ax = -by$

$\letus d = \gcd(a, b),~ a = da',~ b = db'$

$ax = -by \iff da'x = -db'y \iff a'x = -b'y \overset{(\star)}{\iff}
\begin{cases}
    x = b'k \\
    y = -a'k
\end{cases},\ k \in \Z$

$(\star) \gcd(a', b') = 1 \implies a' \mid y,~ b' \mid x \implies x = b'k,~ k\in \Z \implies y = -a'k$

Теперь найдём какое-то решение исходного уравнения, вспомнив о линейном представлении $\gcd$:

$\gcd(a, b) = d = ax_0 + by_0 \implies c = dc' = a(c'x_0) + b(c'y_0)$

Таким образом, решение исходного уравнения: $\{(c'x_0 + b'k,~ c'y_0 -a'k) \mid k \in \Z\}$, где:

$x_0, y_0$ - коэффициенты при $a, b$ в линейном представлении $\gcd(a, b)$,

$a' = \frac{a}{\gcd(a, b)}$, $b' = \frac{b}{\gcd(a, b)}$, $c' = \frac{c}{\gcd(a, b)}$

\lstinputlisting[style=supercpp]{code/linear-diophantine-equation.cpp}
