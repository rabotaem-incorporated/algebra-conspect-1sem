\subsection{Взаимно простые числа}

\begin{defn}
    Числа $a$ и $b$ называются взаимно простыми, если $\gcd(a, b) = 1$.
\end{defn}

\begin{theorem-non}~
    \begin{enumerate}
        \item $\letus a, b \in \Z$, тогда $a \bot b \iff \exists m, n \in \Z: am + bn = 1$.
        
        \item $a_1 \bot b, a_2 \bot b \implies a_1a_2 \bot b$.
        
        \item $a_1, \ldots, a_m, b_1, \ldots, b_n \in \Z$ и $\forall i, j: a_i \bot b_j \implies a_1 \ldots a_m \bot b_1 \ldots b_n$.
        
        \item $a \mid bc,~a \bot b \implies a \mid c$.
        
        \item $ax \equivm{m} ay,~a \bot m \implies x \equivm{m} y$.
        
        \item $\gcd(a, b) = d \implies a = da',~b = db', a' \bot b'$.
    \end{enumerate}
\end{theorem-non}

\begin{proof}
    \begin{enumerate}
        \item $m$ и $n$ существуют согласно линейному представлению НОД.
        
        $d = \gcd(a, b), d \mid a, d \mid b \implies d \mid (am + bn) = 1 \implies d \mid 1 \implies d = 1$.
        
        \item $1 = a_1m_1 + bn_1, 1 = a_2m_2 + bn_2 \implies 1 = a_1a_2(m_1m_2) + b(a_1m_1n_2 + a_2m_2n_1+bn_1n_2) \implies a_1a_2 \bot b$.
        
        \item $a_1 \bot b, \ldots,~a_n \bot b \implies a_1 \ldots a_n \bot b$
        
        $a_1 \ldots a_n \bot b_1, \ldots,~a_1 \ldots a_n \bot b_n \implies a_1 \ldots a_n \bot b_1 \ldots b_n$
        
        \item $1 = am + bn, c = acm + bcn$
        
        $a \mid acm,~a \mid bcn \implies a \mid c$.
        
        \item $m \mid (ax - ay), a \bot m \implies m \mid (x - y) \implies x \equivm{m} y$.
        
        \item $d \mid a, d \mid b \implies a = da',~b=db': a', b' \in \Z$
        
        $d = am + bn,~m, n \in \Z$
        
        $d = 0 \implies a' = b' = 0 = da'm + db'm$
        
        $d \neq 0 \implies 1 = a'm + b'n \implies a' \bot b'$.
    \end{enumerate}
\end{proof}