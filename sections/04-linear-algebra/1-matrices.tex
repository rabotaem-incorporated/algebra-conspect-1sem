\subsection{Матрицы}

\begin{defn}
    $R$ --- кольцо, $m, n \in \N$

    Матрица $m \times n$ над кольцом $R$ --- прямоугольная таблица

    $A = \begin{pmatrix}
        a_{11} & a_{12} & \dots & a_{1n} \\
        a_{21} & a_{22} & \dots & a_{2n} \\
        \vdots & \vdots & \ddots & \vdots \\
        a_{m1} & a_{m2} & \dots & a_{mn}
    \end{pmatrix}$, где $a_{ij} \in R$

    Есть краткая запись $A = (a_{ij})_{i = 1, \ldots, m;~j = 1, \ldots, n} = (a_{ij})$
\end{defn}

\begin{defn}
    Множество матриц $m \times n$ над кольцом $R$ обозначается как $M_{m, n}(R)$

    Так же обозначают, как: $R^{m \times n}$, $M(m, n, R)$, $M_{m \times n}(R)$
\end{defn}

Пусть $A, B \in M_{m, n}(R)$ --- матрицы. $A = (a_{ij})$, $B = (b_{ij})$

Их суммой называется матрица $C = (c_{ij})$, где $c_{ij} = a_{ij} + b_{ij}$.

Пусть $A = (a_{ij}) \in M_{m, n}(R)$, $B = (b_{ij}) \in M_{n, p}(R)$

Их произведением называется матрица $C = (c_{ij}) \in M_{m, p}(R)$, где $c_{ij} = \sum\limits_{k = 1}^n a_{ik} b_{kj}$

Пусть $c \in R$, $A \in M_{m, n}(R)$

Тогда $c \cdot A = (c \cdot a_{ij}) \in M_{m, n}(R)$

\begin{notice}
    По умолчанию $R$ --- коммутативное кольцо
\end{notice}

\begin{defn}
    Транспонированная матрица $A = (a_{ij}) \in M_{m, n}(R)$ --- матрица $B = (b_{ij}) \in M_{n, m}(R)$, где $b_{ij} = a_{ji}$

    Обозначается как $A^T$
\end{defn}

\begin{example}
    $\begin{pmatrix}
        2 & 0 & -3 \\
        1 & 5 & 4
    \end{pmatrix}^T = \begin{pmatrix}
        2 & 1 \\
        0 & 5 \\
        -3 & 4
    \end{pmatrix}$
\end{example}

\begin{defn}
    Матрица $A = (a_{ij}) \in M_{m, n}(R)$ --- квадратная, если $m = n$

    Обозначается как $A \in M_{n}(R)$
\end{defn}

\begin{theorem}[Свойства операций над матрицами]~

    \begin{enumerate}
        \item $A + (B + C) = (A + B) + C$
        
        \item $0 = (0)$, тогда $A + 0 = 0 + A = A$
        
        \item Для любой $A$ есть $-A$, такая что $A + (-A) = (-A) + A = 0$
        
        \item $A + B = B + A$
        
        \item $(AB)C = A(BC)$, нужно чтобы $A \in M_{m, n}(R)$, $B \in M_{n, p}(R)$, $C \in M_{p, q}(R)$
        
        Обе матрицы принадлежат $M_{m, q}(R)$

        \item $A(B + C) = AB + AC$
        
        \item $(B + C)A = BA + CA$
    
        \item $(\lambda + \mu) A = \lambda A + \mu A,~\lambda, \mu \in R$
        
        \item $\lambda(A + B) = \lambda A + \lambda B,~\lambda \in R$
        
        \item $(\lambda A) B = \lambda (A B) = A(\lambda B),~\lambda \in R$
        
        \item $(\lambda \mu) A = \lambda (\mu A),~\lambda, \mu \in R$
        
        \item $(A + B)^T = A^T + B^T$
        
        \item $(AB)^T = B^T A^T$
    \end{enumerate}
\end{theorem}

\begin{defn}
    Пусть $n \in \N$. Единичной матрицой порядка $n$ называется:

    $E_n = \begin{pmatrix}
        1 & 0 & \ldots & 0 \\
        0 & 1 & \ldots & 0 \\
        \vdots & \vdots & \ddots & \vdots \\
        0 & 0 & \ldots & 1  
    \end{pmatrix} \in M_{n}(R)$

    Как кратко обозначить: $E_n = (\delta_{ij})$, где $\delta_{ij} = \begin{cases}
        1, & i = j \\
        0, & i \neq j
    \end{cases}$ --- символ Кронекера
\end{defn}

\begin{theorem-non}

    Пусть $A \in M_{m, n}(R)$.

    Тогда $E_m A = A E_n = A$
\end{theorem-non}

\begin{proof}

    $E_m A = (b_{ij}),~ A = (a_{ij})$

    $b_{ij} = \sum\limits_{k = 1}^m \delta_{ik} a_{kj} = a_{ij}$

    То есть $E_m A = A$

    $E_n A^T = A^T \implies (E_n A^T)^T = (A^T)^T \implies (A^T)^T E_n^T = (A^T)^T \implies A E_n = A$
\end{proof}

\begin{follow}
    $M_{n}(R)$ --- кольцо, где $E_n$ --- нейтральный элемент по умножению

    Называют кольцом квадратных матриц порядка $n$.
\end{follow}

\begin{notice}
    Кольцо не обязательно коммутативное при $n \geq 2$

    $A = \begin{pmatrix}
        0 & 1 \\
        0 & 0
    \end{pmatrix} \cdot \begin{pmatrix}
        0 & 0 \\
        1 & 0
    \end{pmatrix} = \begin{pmatrix}
        1 & 0 \\
        0 & 0
    \end{pmatrix}$
    
    $B = \begin{pmatrix}
        0 & 0 \\
        1 & 0
    \end{pmatrix} \cdot \begin{pmatrix}
        0 & 1 \\
        0 & 0
    \end{pmatrix} = \begin{pmatrix}
        0 & 0 \\
        0 & 1
    \end{pmatrix}$

    $A \neq B$
\end{notice}

\begin{notice}
    $M_1(R) \cong R$
\end{notice}

\begin{defn}
    $GL_n(R) = M_n(R)^* = \{A \in M_n(R) \mid \exists B \in M_n(R),~ AB = BA = E_n\}$

    Такая $B$ единственная и называется обратной к $A$, обозначается $A^{-1}$
\end{defn}

\begin{theorem-non}~

    \begin{enumerate}
        \item $E_n \in GL_n(R),~ E_n^{-1} = E_n$
        
        \item $A_1, \ldots, A_k \in GL_n(R) \implies \prod\limits_{i = 1}^k A_i \in GL_n(R),~ (A_1 \ldots A_k)^{-1} = A_k^{-1} \ldots A_1^{-1}$
        
        \item $A \in GL_n(R) \implies A^T \in GL_n(R),~ (A^T)^{-1} = (A^{-1})^T$
    \end{enumerate}
\end{theorem-non}

\begin{proof}
    \begin{enumerate}
        \item $E_n E_n = E_n E_n = E_n$
        
        \item $(A_1 \ldots A_k)(A_k^{-1} \ldots A_1^{-1}) = A_1 \ldots A_{k-1} (A_k A_k^{-1}) \ldots A_1^{-1} = A_1 \ldots A_{k - 1} A_{k - 1}^{-1} \ldots A_1^{-1} = A_1 A_1^{-1} = E_n$
        
        $(A_k^{-1} \ldots A_1^{-1})(A_1 \ldots A_k) = \ldots = A_k^{-1} A_k = E_n$

        \item $(A^T \cdot (A^T)^{-1}) = (A^{-1} \cdot A)^T = E_n^T = E_n$
        
        $((A^T)^{-1} \cdot A^T) = (A \cdot A^{-1})^T = E_n^T = E_n$
    \end{enumerate}
\end{proof}

\begin{defn}
    Матричная единица --- это матрица, где все элементы нулевые, кроме одного, который равен единице.

    Обозначается как $e_{ij}$.
\end{defn}

\begin{notice}
    $A = (a_{ij}) = \sum\limits_{i, j} a_{ij} e_{ij}$
\end{notice}