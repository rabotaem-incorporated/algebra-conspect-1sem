\subsection{Перестановки}

\begin{defn}
    $M$ --- множество. Перестановкой $M$ называется биекция на себя.

    $S(M) = \{ \text{перестановка } M \}$

    $S(M) \times S(M) \to S(M)$

    $(g, f) \mapsto g \circ f$
\end{defn}

\begin{theorem-non}
    $(S(M), \circ)$ --- группа.
\end{theorem-non}

\begin{proof}
    \begin{enumerate}
        \item Ассоциативность очевидна.
        
        \item $id_M$ --- нейтральный элемент.
        
        \item $f \in S(M) \implies f^{-1} \in S(M)$ --- обратный элемент.
    \end{enumerate}
\end{proof}

\begin{defn}
    $S_n$ --- симметрическая группа степени $n$ (группа перестановок $n$-элементного множества)
\end{defn}

\begin{notice}
    $|S_n| = n!$
\end{notice}

\begin{example}
    $S_3 = \{ (1, 2, 3), (1, 3, 2), (2, 1, 3), (2, 3, 1), (3, 1, 2), (3, 2, 1) \}$
\end{example}

\begin{defn}
    Циклом $(i_1, i_2, \ldots, i_k)$ называется $\sigma \in S_n$ такая что $\sigma(i_1) = i_2, \sigma(i_2) = i_3, \ldots, \sigma(i_{k - 1}) = i_k, \sigma(i_k) = i_1$, а так же $\sigma(i_j) = i_j$ для всех $j \notin \{ 1, 2, \ldots, k \}$.

    $k \geq 2$ --- длина цикла.
\end{defn}

\begin{defn}
    Циклы $(i_1, i_2, \ldots, i_k)$ и $(j_1, j_2, \ldots, j_l)$ называются независимыми, если $i_r \neq j_s \forall r,s$
\end{defn}

\begin{theorem-non}
    Любая перестановка является произведением нескольких попарно независимых циклов.
\end{theorem-non}