\subsection{Перестановки}

\begin{defn}
    $M$ --- множество. \emph{Перестановкой} $M$ называется биекция на себя.

    $S(M) = \{ \text{перестановка } M \}$

    $S(M) \times S(M) \to S(M)$

    $(g, f) \mapsto g \circ f$
\end{defn}

\begin{theorem-non}
    $(S(M), \circ)$ --- группа.
\end{theorem-non}

\begin{proof}
    \begin{enumerate}
        \item Ассоциативность очевидна.
        
        \item $id_M$ --- нейтральный элемент.
        
        \item $f \in S(M) \implies f^{-1} \in S(M)$ --- обратный элемент.
    \end{enumerate}
\end{proof}

\begin{defn}
    $S_n$ --- симметрическая группа степени $n$ (\emph{группа перестановок $n$-элементного множества})
\end{defn}

\begin{notice}
    $|S_n| = n!$
\end{notice}

\begin{example}
    $S_3 = \{ (1, 2, 3), (1, 3, 2), (2, 1, 3), (2, 3, 1), (3, 1, 2), (3, 2, 1) \}$
\end{example}

\begin{defn}
    \emph{Циклом} $(i_1, i_2, \ldots, i_k)$ называется $\sigma \in S_n$ такая что $$\sigma(i_1) = i_2, \sigma(i_2) = i_3, \ldots, \sigma(i_{k - 1}) = i_k, \sigma(i_k) = i_1,$$ а так же $\sigma(i_j) = i_j$ для всех $j \notin \{ 1, 2, \ldots, k \}$.

    $k \geq 2$ --- \emph{длина цикла}.
\end{defn}

\begin{defn}
    Циклы $(i_1, i_2, \ldots, i_k)$ и $(j_1, j_2, \ldots, j_l)$ называются \emph{независимыми}, если $\forall r,s : i_r \neq j_s$
\end{defn}

\begin{theorem-non}
    Любая перестановка является произведением нескольких попарно независимых циклов.
\end{theorem-non}

\begin{proof}
    $i, \sigma(i), \sigma(\sigma(i)), \ldots$ все различны, так как $\sigma$ --- биекция, значит это --- независимый цикл.
\end{proof}

\begin{defn}
    Цикл длины $2$ называется \emph{транспозицией}.
\end{defn}

\begin{defn}
    Транспозиция $(i, i + 1)$ назывется \emph{элементарной транспозицией}.
\end{defn}

\begin{theorem-non}
    Любой цикл $(i_1, i_2, \ldots, i_n)$ раскладывается в произведение транспозиций $(i_1, i_2) \cdot (i_2, i_3) \cdot \ldots \cdot (i_{n-1}, i_n)$
\end{theorem-non}

\begin{exerc}
    Любая перестановка раскладывается в произведение элементраных транспозиций.
\end{exerc}

\begin{defn}
    $(i, j), ~i < j$ --- \emph{инверсия}, если $\sigma(i) > \sigma(j)$
\end{defn}

\begin{defn}
    $\Inv(\sigma)$ --- \emph{число инверсий} перестановки $\sigma$
\end{defn}

\begin{defn}
    \emph{Четность перестановки} --- четность числа инверсий в ней.
\end{defn}

\begin{defn}
    \emph{Знак перестановки} ---  $\sgn(\sigma) = \begin{cases}
        1, ~\text{ если перестановка четная} \\
        -1, ~\text{ если перестановка нечетная}\\
    \end{cases}$
\end{defn}

\begin{lemma}
    Если перестановку умножить справа на транспозицию, то ее знак поменяется на противоположный, то есть $\sgn(\sigma \circ (i, j)) = -\sgn(\sigma)$.
\end{lemma}

\begin{proof}
    Четность числа инверсий с участием $\sigma(i)$ и $\sigma(j)$ не изменится, так как все элементы между $i$ и $j$ поменяют число инверсий четное число раз. Соответственно, изменится лишь инверсия между $i$ и $j$.
\end{proof}

\begin{follow}
    Четность перестановки равна четности количества транспозиций в ее разложении.
\end{follow}

\begin{proof}
    $\sgn((i_1, j_1) (i_2, j_2) \ldots (i_k, j_k)) = (-1)^k$
\end{proof}

\begin{follow} 
    Пусть $\sigma, \tau \in S_n$, тогда:

    $\sgn(\sigma \tau) = \sgn(\sigma) \sgn(\tau)$

    $\sgn(\sigma ^{-1}) = \sgn(\sigma)$
\end{follow}

\begin{defn}
    \emph{Множество четных перестановок} $A_n = \{\sigma \in S_n \mid \sgn(\sigma) = 1\}$ --- подгруппа $S_n$
\end{defn}

\begin{theorem-non}
    Пусть $n \geq 2$, тогда $|A_n| = \frac{n!}{2}$ 
\end{theorem-non}

\begin{proof}
    Рассмотрим 

    $\varphi: A_n \to S_n \setminus A_n$
    
    $\sigma \mapsto \sigma \circ (1, 2)$ --- из четной перестановки получаем нечетную

    $\psi: S_n \setminus A_n \to A_n$

    $\sigma \mapsto \sigma \circ (1, 2)$ --- из нечетной перестановки получаем четную

    $\varphi = \psi^{-1} \implies \varphi$ --- биекция $\implies |A_n| = |S_n \setminus A_n| = \frac{n!}{2}$
\end{proof}