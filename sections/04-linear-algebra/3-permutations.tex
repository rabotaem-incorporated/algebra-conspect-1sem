\subsection{Перестановки}

\begin{defn}
    $M$ --- множество. Перестановкой $M$ называется биекция на себя.

    $S(M) = \{ \text{перестановка } M \}$

    $S(M) \times S(M) \to S(M)$

    $(g, f) \mapsto g \circ f$
\end{defn}

\begin{theorem-non}
    $(S(M), \circ)$ --- группа.
\end{theorem-non}

\begin{proof}
    \begin{enumerate}
        \item Ассоциативность очевидна.
        
        \item $id_M$ --- нейтральный элемент.
        
        \item $f \in S(M) \implies f^{-1} \in S(M)$ --- обратный элемент.
    \end{enumerate}
\end{proof}

\begin{defn}
    $S_n$ --- симметрическая группа степени $n$ (группа перестановок $n$-элементного множества)
\end{defn}

\begin{notice}
    $|S_n| = n!$
\end{notice}

\begin{example}
    $S_3 = \{ (1, 2, 3), (1, 3, 2), (2, 1, 3), (2, 3, 1), (3, 1, 2), (3, 2, 1) \}$
\end{example}

\begin{defn}
    Циклом $(i_1, i_2, \ldots, i_k)$ называется $\sigma \in S_n$ такая что $\sigma(i_1) = i_2, \sigma(i_2) = i_3, \ldots, \sigma(i_{k - 1}) = i_k, \sigma(i_k) = i_1$, а так же $\sigma(i_j) = i_j$ для всех $j \notin \{ 1, 2, \ldots, k \}$.

    $k \geq 2$ --- длина цикла.
\end{defn}

\begin{defn}
    Циклы $(i_1, i_2, \ldots, i_k)$ и $(j_1, j_2, \ldots, j_l)$ называются независимыми, если $i_r \neq j_s \forall r,s$
\end{defn}

\begin{theorem-non}
    Любая перестановка является произведением нескольких попарно независимых циклов.
\end{theorem-non}

\begin{proof}
    $i, \sigma(i), \sigma(\sigma(i)), \ldots$ все различны, так как $\sigma$ - биекция, значит это - независимый цикл.
\end{proof}

\begin{defn}
    Цикл длины 2 называется транспозицией
\end{defn}

\begin{defn}
    Транспозиция $(i, i + 1)$ назывется элементарной
\end{defn}

\begin{theorem-non}
    Либой цикл $(i_1, i_2, \ldots, i_n)$ раскладывается в произведение транспозиций $(i_1, i_2) \cdot (i_2, i_3) \cdot \ldots \cdot (i_{n-1}, i_n)$
\end{theorem-non}

\begin{follow}
    Любая перестановка раскладывается в произведение элементраных транспозиций - Упражнение.
\end{follow}

\begin{defn}
    $(i, j), ~i < j$ - инверсия, если $\sigma(i) > \sigma(j)$

    $Inv(\sigma)$ - число инверсий перестановки $\sigma$
\end{defn}

\begin{defn}
    Четность перестановки - четность числа инверсий в ней.

    Знак перестановки -  $\sgn(\sigma) = \begin{cases}
        1, ~\text{ если перестановка четная} \\
        -1, ~\text{ если перестановка нечетная}\\
    \end{cases}$
\end{defn}

\begin{lemma}
    Если перестановку умножить справа на транспозицию, то ее знак поменяется на противоположный
\end{lemma}

\begin{proof}
    Доказывается пристальным рассмотрением того, что происходит, когда в перестановке два элемента меняются местами.
\end{proof}

\begin{follow}
    Четность перестановки равна четности количества транспозиций в ее разложении
\end{follow}

\begin{follow}~

    $\sgn(\sigma \tau) = \sgn(\sigma) \sgn(\tau)$

    $\sgn(\sigma ^{-1}) = \sgn(\sigma)$
\end{follow}

$A_n = \{\sigma \in S_n \mid \sgn(\sigma) = 1\}$ - подгруппа $S_n$

\begin{theorem-non}
    Пусть $n \ge 2$, тогда $|A_n| = \frac{n!}{2}$ 
\end{theorem-non}

\begin{proof}
    Рассмотрим 

    $\varphi: A_n \mapsto S_n \setminus A_n$
    
    $\sigma \to \sigma(1 ~2)$

    $\psi: S_n \setminus A_n \mapsto A_n$

    $\sigma \to \sigma(1 ~2)$

    $\varphi = \psi^-1 \implies \varphi$ - биекция $\implies |A_n| = |S_n \setminus A_n| = \frac{n!}{2}$
\end{proof}