\subsection{Определители}

Мы знаем, что матрицы тесно связаны с системами линейных уравнений и мы хотим знать, когда системы разрешимы единственным образом, когда не имеют решений, и когда имеют бесконечно много решений.

\begin{defn}
    Системы линейных уравнений подразделяются на:

    \begin{itemize}
        \item \emph{Несовместные} --- не имеют решений.
        \item \emph{Совместные} --- имеют решения.
        \begin{itemize}
            \item \emph{Определенные} --- имеют единственное решение.
            \item \emph{Неопределенные} --- имеют бесконечно много решений.
        \end{itemize}
    \end{itemize}
\end{defn}

\begin{defn}
    $A \in M_n(R),~ R$ --- коммутативное кольцо. 
    
    \emph{Определителем} матрицы $A$ называется: 
    $$\det(A) = |A| = \sum\limits_{\sigma \in S_n} \sgn(\sigma)\prod\limits_{i = 1}^n a_{i\sigma(i)}$$
\end{defn}


\begin{notice}
    *Мотивация*: Определитель это функция $\det: M_{n}(K) \mapsto K$, такая что:

    $\det(A) \neq 0 \iff Ax = b$ --- совместная определенная для всех $b$.
\end{notice}

\begin{theorem-non}
    $\det(A) = \det(A^T)$
\end{theorem-non}

\begin{proof}
    \begin{gather*}
        |A^T| = \sum\limits_{\sigma \in S_n} \sgn \sigma \cdot \prod\limits_{i = 1}^n a_{\sigma(i)i} = \\
        \sum\limits_{\sigma \in S_n} \sgn \sigma \cdot \prod\limits_{i = 1}^n a_{i\sigma^{-1}(i)} = \\
        \sum\limits_{\sigma \in S_n} \sgn \sigma^{-1} \cdot \prod\limits_{i = 1}^n a_{i\sigma^{-1}(i)} = \\
        \sum\limits_{\sigma \in S_n} \sgn \sigma \cdot \prod\limits_{i = 1}^n a_{i\sigma(i)} = |A|
    \end{gather*}
\end{proof}