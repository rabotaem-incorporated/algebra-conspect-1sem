\subsection{Определители}

Мы знаем, что матрицы тесно связаны с системами линейных уравнений и мы хотим знать, когда системы разрешимы единственным образом, когда не имеют решений, и когда имеют бесконечно много решений.

\begin{defn}
    Системы линейных уравнений подразделяются на:

    \begin{itemize}
        \item \emph{Несовместные} --- не имеют решений.
        \item \emph{Совместные} --- имеют решения.
        \begin{itemize}
            \item \emph{Определенные} --- имеют единственное решение.
            \item \emph{Неопределенные} --- имеют бесконечно много решений.
        \end{itemize}
    \end{itemize}
\end{defn}

\begin{defn}
    $A \in M_n(R),~ R$ --- коммутативное кольцо. 
    
    \emph{Определителем} матрицы $A$ называется: 
    $$\det(A) = |A| = \sum\limits_{\sigma \in S_n} \sgn(\sigma)\prod\limits_{i = 1}^n a_{i\sigma(i)}$$
\end{defn}


\begin{prop}~

    1. $\det(A) = \det(A^T)$

    2. $A = \begin{bmatrix}
        A_1 \\
        \vdots \\
        A_i' + A_i'' \\
        \vdots \\
        A_n
    \end{bmatrix}$,~ $\det(A) = \begin{vmatrix}
        A_1 \\
        \vdots \\
        A_i' \\
        \vdots \\
        A_n
    \end{vmatrix} + \begin{vmatrix}
        A_1 \\
        \vdots \\
        A_i'' \\
        \vdots \\
        A_n
    \end{vmatrix}$

    3. $A = \begin{bmatrix}
        A_1 \\
        \vdots \\
        \alpha A_i \\
        \vdots \\
        A_n
    \end{bmatrix}$,~ $\det(A) = \alpha \begin{vmatrix}
        A_1 \\
        \vdots \\
        A_i \\
        \vdots \\
        A_n
    \end{vmatrix}$

    4. $\det(\alpha A) = \alpha^n \det(A)$

    5. $A = \begin{bmatrix}
        A_1 \\
        \vdots \\
        A_n
    \end{bmatrix}$,~ $A_i = A_j \implies \det(A) = 0$

    6. $A = \begin{bmatrix}
        A_1 \\
        \vdots \\
        A_i \\
        \vdots \\
        A_j \\
        \vdots \\
        A_n
    \end{bmatrix}$,~ $B = \begin{bmatrix}
        A_1 \\
        \vdots \\
        A_j \\
        \vdots \\
        A_i \\
        \vdots \\
        A_n
    \end{bmatrix}$,~ $\det(A) = -\det(B)$

    7. $A = \begin{bmatrix}
        A_1 \\
        \vdots \\
        A_i \\
        \vdots \\
        A_n
    \end{bmatrix}$,~ $B = \begin{bmatrix}
        A_1 \\
        \vdots \\
        A_i + \alpha A_j \\
        \vdots \\
        A_n
    \end{bmatrix}$,~ $\det(A) = \det(B)$
\end{prop}

\begin{proof}~

    1. \begin{gather*}
        |A^T| = \sum\limits_{\sigma \in S_n} \sgn \sigma \cdot \prod\limits_{i = 1}^n a_{\sigma(i)i} = \\
        \sum\limits_{\sigma \in S_n} \sgn \sigma \cdot \prod\limits_{i = 1}^n a_{i\sigma^{-1}(i)} = \\
        \sum\limits_{\sigma \in S_n} \sgn \sigma^{-1} \cdot \prod\limits_{i = 1}^n a_{i\sigma^{-1}(i)} = \\
        \sum\limits_{\sigma \in S_n} \sgn \sigma \cdot \prod\limits_{i = 1}^n a_{i\sigma(i)} = |A|
    \end{gather*}

    \TODO[Дописать]
\end{proof}

\subsection{Дальнейшие свойства определителя}

\begin{prop}~

    $A = \begin{bmatrix}
        X & Y \\
        0 & Z
    \end{bmatrix}$,~ $\det(A) = \det(X) \det(Z)$
\end{prop}

\begin{proof}~

    $\det(A) = \sum_{\sigma \in S_n} \sgn(\sigma) \prod_i a_{i, \sigma(i)} = \sum_{\sigma \in S_n} A_{\sigma}$
    
    $A_{\sigma} $ не включает элементов из нижней левой нулевой матрицы, значит $\sigma(\{n + 1, n + 2, \ldots, m\}) = \{n + 1, n + 2, \ldots, m\}$ и $\sigma(\{1, 2, \ldots, n\}) = \{1, 2, \ldots, n\}$

    А тогда $\sigma = \gamma \delta, ~ \gamma \in S_n, ~ \delta \in S_m, ~ A_{\sigma} = \sgn(\gamma) X_{\gamma} \sgn(\delta) Z_\delta$
\end{proof}

\begin{prop}~

    $A, B \in M_{n \times n}(K), ~ \det(AB) = \det(A) \det(B)$
\end{prop}

\begin{proof}
    Элементарными преобразованиями можно и $A$ и $B$ привести к диагональному виду и тогда определители посчитаются легко
\end{proof}

\begin{defn}
    Минором $M_{ij}$ матрицы $A$ порядка $1$ называется определитель $A_{kl}$, где $i \neq k, j \neq l$.

    Алгебраическим дополнением $A_{ij}$ называется $(-1)^{i + j} M_{ij}$.
\end{defn}

\begin{lemma}
    Об определители квадратной матрицы с почти нулевой строкой.

    $A \in M_{n \times n}(K) ~ 1 \le i_0, j_0 \le n, ~ a_{i_0j} = 0$ при $j \neq j_0$

    Тогда $\det(A) = a_{i_0j_0}\det(A_{i_0j_0})$
\end{lemma}

\begin{proof}
    $\sigma(i_0) = j_0$ и потом посмотреть просто как знак перестановки меняется
\end{proof}

\begin{theorem}
    Разложение определителя по $k$-ой строке

    $A \in M_{n \times n}(K),~ \det(A) = \sum_{i = 1}^n a_{ki}A_{ki}$

    Тогда $\det(A) = \prod\limits_{i = 1}^n a_{ii}$
\end{theorem}

\begin{proof}
    Разложим $k$-ую строку в сумму строк вида $\begin{bmatrix}
        0 & \ldots & 0 & a_{ki} & 0 & \ldots & 0
    \end{bmatrix}$ и применим лемму
\end{proof}

\begin{follow}
    Разложение по столбцу ничем не отличается
\end{follow}

\begin{follow}
    Матрица обратима $\iff$ $\det(A) \neq 0$
\end{follow}

\begin{lemma}
    Пусть $1 \le i \neq j \le n$

    Тогда $\sum\limits_{k = 1}^n a_{ik}A_{jk} = 0$
\end{lemma}

\begin{proof}
    Это выражение это определитель матрицы $A' = \begin{bmatrix}
        A_1 \\
        \vdots \\
        A_j \\
        \vdots \\
        A_j \\
        \vdots \\
        A_n
    \end{bmatrix}$
\end{proof}

\begin{follow}~
    
    $\sum\limits_{k = 1}^n a_{ik}A_{jk} = \begin{cases}
        \det(A), ~ i = j \\
        0, ~ i \neq j
    \end{cases}$
    
\end{follow}

\begin{defn}
    Взаимная матрица $\tilde{A}$ это матрица $(A_{ij})_{i,j}^T$
\end{defn}

\begin{follow}~

    $A\tilde{A} = \tilde{A} = \det(A) I_n$
    
    $A^{-1} = \det(A)^{-1}\tilde{A}$
\end{follow}

\begin{theorem}
    (Крамера)

    $A \in M_n(K)$, тогда
    
    $A$ - совместная определенная $\iff$ $\det(A) \neq 0$
\end{theorem}

\begin{proof}~

    $2 \implies 1$ 
    
    $A \in GL_n(K)$, тогда

    $AX = b \iff X = A^{-1}b$

    $1 \implies 2$

    Приведем методом Гаусса к ступенчатому виду. Таким образом можно считать, что $(A|b)$ ступенчатая.
    Посмотрим на последнюю ступеньку. Если ведущий елемент последней ненулевой строки находится в последнем столбце, то система несовместная, что невозможно.
    В противном случае все ступеньки начинаются левее верикальной черты и нет свободных неизвестных. Поскольку система определенная, ширина всех ступенек равна $1$.
    Таким образом 
    
    $A = \begin{bmatrix}
        a_1& * & \ldots & * \\
        0 & a_2 & \ldots & * \\
        \vdots & \vdots & \ddots & \vdots \\
        0 & 0 & \ldots & a_n
    \end{bmatrix}$, где $a_i \neq 0$. 
    
    Тогда $\det(A) = \prod\limits_{i = 1}^n a_i \neq 0$
\end{proof}