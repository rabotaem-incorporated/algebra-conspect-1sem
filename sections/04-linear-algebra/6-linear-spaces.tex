\subsection{Линейные пространства}

\subsubsection{Основные определения}

\begin{defn}~
    
    Пусть $K$ - поле. Говорят, что задано линейное пространство над полем $K$, если заданы

    \begin{enumerate}
        \item множество $V$ 
        \item операция $V \times V \to V$ - сложение 
        \item операция $K \times V \to V$ - умножение на скаляр 
    \end{enumerate}

    Такие что выполняются следующие аксиомы:

    Пусть $a, b \in K, ~ A, B \in V$, тогда

    $(V, +)$ - абелева группа
    \begin{enumerate}
        \item $a(A + B) = aA + aB$
        \item $(a + b)A = aA + bA$
        \item $(ab)A = a(bA)$
        \item $1A = A$
    \end{enumerate}

\end{defn}

\begin{example}~

    \begin{enumerate}
        \item $K^n = M(n, 1, K)$ - арифметическое $n$-мерное пространство над полем $K$.
        \item $V = \left\{\begin{bmatrix}
            a \\
            b \\
            c
        \end{bmatrix} ~ \middle| ~ a + b + c = 0\right\}$
        \item $V = K[x]$
        \item $V = C[0, 1]$
        \item $K = \R, ~ V = \R_+$

        $\bigoplus: ~ v_1 \bigoplus v_2 = v_1v_2$
    
        $\bigotimes: ~ a \bigotimes v = v^a$
        \item $K = \F_2 = \{0, 1\}, ~ M - $ множество, $V = 2^M$

        $\bigoplus: ~ v_1 \bigoplus v_2 = v_1\triangle v_2$
    
        $\bigotimes: ~ a \bigotimes v = \begin{cases}
            1 \bigotimes v = v \\
            0 \bigotimes v = \emptyset
        \end{cases}$
    \end{enumerate}
    
\end{example}

\begin{lemma}~

    $0 \cdot v = 0$

    $a \cdot 0 = 0$

    $av = 0 \iff a = 0 \lor v = 0$

    $(-1) \cdot v = -v$
\end{lemma}

\begin{proof}~

    $0 \cdot v = (0 + 0) \cdot v = 0 \cdot v + 0 \cdot v \implies 0 \cdot v = 0$

    $a \cdot 0 = a \cdot (0 + 0) = a \cdot 0 + a \cdot 0 \implies a \cdot 0 = 0$

    $a \neq 0 \implies a^{-1}av = v = 0 \implies v = 0$

    $v + (-1) \cdot v = 1 \cdot v + (-1) \cdot v = 0 \cdot v = 0 = v + -v \implies (-1) \cdot v = -v$
\end{proof}

\subsubsection*{Системы образующих. Линейные подпространства}

\begin{defn}~

    $a_1, a_2, \ldots, a_n \in K, ~ v_1, v_2, \ldots, v_n \in V$, тогда

    $a_1v_1 + a_2v_2 + \ldots + a_nv_n$ - линейная комбинация векторов $v_1, v_2, \ldots, v_n$ с коэффициентами $a_1, a_2, \ldots, a_n$.

    $Lin(v_1, v_2, \ldots, v_n) = \langle v_1, v_2, \ldots, v_n \rangle = \left\{a_1v_1 + a_2v_2 + \ldots + a_nv_n ~ \middle| ~ a_1, a_2, \ldots, a_n \in K\right\}$ - линефная оболочка $v_1, v_2, \ldots, v_n$.

    Пусть $M \subset V$
    
    $\langle M \rangle = V$, говорят, что $M$ - система/семейство образующих/порождающих пространства $V$.
\end{defn}

\begin{theorem-non}~
    
    $M = \{v_1, v_2, \ldots, v_n\}$
    
    Пусть $v_n$ представляется в виде линейной комбинации $v_1, v_2, \ldots, v_{n-1}$, тогда

    $\langle M \rangle = \langle M \setminus v_n \rangle$
\end{theorem-non}

\begin{proof}~

    $v_n = a_1v_1 + a_2v_2 + \ldots + a_{n-1}v_{n-1}$

    $\forall w \in \langle M \rangle~ w = b_1v_1 + b_2v_2 + \ldots + b_{n-1}v_{n-1} + b_nv_n = (b_1 + b_na_1) v_1 + (b_2 + b_na_2) v_2 + \ldots + (b_{n-1} + b_na_{n-1}) v_{n-1} \in \langle M \setminus v_n \rangle$

    Обратное включение очевидно.
\end{proof}

\begin{defn}
    Пространство $V$ называется конечномерным, если у него есть конечная система образующих.
\end{defn}

\begin{example}~

    \begin{enumerate}
        \item $M(n, m, K) = \langle e_{ij} \rangle$ - пространство размерности $nm$
        \item $R_+$ - пространство размерности 1
        \item $K[x] = \langle x^i \rangle, 0 \le i$ - пространство бесконечномерное 
    \end{enumerate}
\end{example}

\begin{defn}
    $V' \subseteq V$ называется линейным подпространством пространства $V$, если
    \begin{enumerate}
        \item $0 \in V'$
        \item $\forall v_1, v_2 \in V'~ v_1 + v_2 \in V'$
        \item $\forall a \in K, ~ v \in V'~ av \in V'$
    \end{enumerate}
\end{defn}

\begin{theorem}~

    Пусть $V'$ - линейное подпространство пространства $V$, тогда $V'$ - линейное пространство.
\end{theorem}

\begin{proof}~

    Из неочевидного только третья аксиома Абелевой группы.

    $v \in V' \implies -v \in V'$ - верно в силу $(-1)v = -v$
\end{proof}

\begin{defn}
    $V' < V$ - обозначает, что $V'$ - линейное подпространство пространства $V$.
\end{defn}

\begin{lemma}~

    Пусть $V' < V$ и $M \subseteq V'$, тогда $\langle M \rangle \subseteq V'$.
\end{lemma}

\begin{proof}
    Очев.
\end{proof}

\subsubsection*{Линейная зависимость и независимость}

\begin{defn}~

    $v_1, v_2, \ldots, v_n \in V$ называются линейно независимыми семейством (ЛНС), если 
    
    $a_1v_1 + a_2v_2 + \ldots + a_nv_n = 0 \implies a_1 = a_2 = \ldots = a_n = 0$.

    $v_1, v_2, \ldots, v_n \in V$ называются линейно зависимыми семейством (ЛЗС), если существует 
    
    $a_1, a_2, \ldots, a_n \in K$ такие, что $a_1^2 + a_2^2 + \ldots + a_n^2 > 0$ и $a_1v_1 + a_2v_2 + \ldots + a_nv_n = 0$.
    
\end{defn}

\begin{theorem} ~

    $v_1, v_2, \ldots, v_n \in V$, тогда эквивалентны следующие утверждения:

    \begin{enumerate}
        \item $v_1, v_2, \ldots, v_n$ - ЛЗС
        \item $\forall j v_j \in \langle v_1, v_2, \ldots, \widehat{v_{j-1}}, v_j, \ldots, v_n \rangle$
        \item $\exists j v_j \in \langle v_1, v_2, \ldots, v_{j-1} \rangle$
    \end{enumerate}
    
\end{theorem}
\begin{proof}~

    $3) \implies 2)$ - очевидно

    $2) \implies 1)$ - $v_j = a_1v_1 + a_2v_2+\ldots \implies 0 = a_1v_1 + a_2v_2 + \ldots + (-1)v_j + \ldots$
    
    $1) \implies 3)$ - $0 = a_1v_1 + a_2v_2 + \ldots \implies 0 = a_1v_1 + a_2v_2 + \ldots + a_jv_j \implies v_j = \frac{-a_1}{a_j}v_1 + \ldots \in \langle v_1, v_2, \ldots, v_{j-1} \rangle$
    
\end{proof}