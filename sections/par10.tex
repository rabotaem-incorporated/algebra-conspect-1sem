\begin{normalsize}

\section{Китайская теорема об остатках}

\begin{theorem}
    Пусть $(m_1, m_2) = 1$; $a_1, a_2 \in \mathZ$.\\
    \begin{enumerate}
        \item $\exists x \in \mathZ$: 
        $\begin{cases}
            x_0 \equivm{m_1} a_1 \\
            x_0 \equivm{m_2} a_2
        \end{cases}$
        \item $\letus x_0$ удовлетворяет системе выше\\
        Тогда для $x \in \mathZ$: $x$ удовлетворяет системе выше $\iff$ $x \equivm{m_1 m_2} x_0$
    \end{enumerate}
\end{theorem}

\begin{proof}
    \begin{enumerate}
        \item $x_0 = a_1 + km_1 = a_2 + lm_2 \implies km_1 - lm_2 = a_2 - a_1$ --- линейное диофантово уравнение с дыумя неизвестными $k, l$\\
        $(m_1, m_2) = 1 \implies$ у него есть решение $(k_0, l_0)$\\
        $x_0 = a_1 + k_0 m_1$ --- искомое

        \item 
        "$\Rightarrow$"\\
        $x \equivm{m_1m_2} x_0 \implies 
        \begin {cases} 
            x \equivm{m_1} x_0 \\
            x \equivm{m_2} x_0
        \end{cases} \implies
        \begin{cases}
            x \equivm{m_1} a_1 \\
            x \equivm{m_2} a_2
        \end{cases}$\\
        "$\Leftarrow$"\\
        $x$ удовлетворяет системе из теоремы $\implies
        \begin {cases} 
            x \equivm{m_1} x_0 \\
            x \equivm{m_2} x_0
        \end{cases} \implies
        \begin{cases}
            m_1 \mid (x - x_0)\\
            m_2 \mid (x - x_0)
        \end{cases} \implies m_1m_2 \mid (x - x_0)$
    \end{enumerate}
\end{proof}

\begin{defn}
    $\letus R, S$ --- кольца с единицей. Отображение $\varphi: R \to S$ называется изоморфизмом колец, если: $\varphi$ биекция.\\
    \begin{enumerate}
        \item $\forall r_1, r_2: \varphi(r_1 + r_2) = \varphi(r_1) + \varphi(r_2)$
        \item $\forall r_1, r_2: \varphi(r_1 r_2) = \varphi(r_1) \varphi(r_2)$
    \end{enumerate}
\end{defn}

\begin{theorem-non}
    $\letus (m_1, m_2) = 1$ \\
    Тогда существует изоморфизм \\
    $\mathZ/m_1m_2\mathZ \to \mathZ/m_1\mathZ \times \mathZ/m_2\mathZ$\\
    $[a]_{m_1m_2} \mapsto ([a]_{m_1}, [a]_{m_2})$
\end{theorem-non}

\begin{proof}
    Проверим корректность:
    $\letus [a]_{m_1m_2} = [a']_{m_1m_2} \implies$\\
    $a \equivm{m_1m_2} a' \implies$\\
    $\begin{cases}
        a \equivm{m_1} a' \\
        a \equivm{m_2} a'
    \end{cases} \implies$
    $([a]_{m_1}, [a]_{m_2}) = ([a']_{m_1}, [a']_{m_2})$\\
    $\varphi([a]_{m_1m_2} + [b]_{m_1m_2}) = \varphi([a + b]_{m_1m_2}) = \varphi([a+b]_{m_1}, [a+b]_{m_2}) = $\\
    $\varphi([a]_{m_1} + [a]_{m_2}) + \varphi([b]_{m_1} + [b]_{m_2}) = \varphi([a]_{m_1m_2}) + \varphi([b]_{m_1m_2})$\\
    Для умножения аналогично.\\
    Проверим сюръективность $\varphi$\\
    $X = ([a_1]_{m_1}, [a_2]_{m_2})$\\
    По китайской теореме об остатках $\exists a \in \mathZ: 
    \begin{cases}
        a \equivm{m_1} a_1 \\
        a \equivm{m_2} a_2
    \end{cases}$\\

    про сюръективность $\implies$ биекция:\\
    $|Y| = |Z| < \infty$\\
    $\varphi: Y \to Z$\\
    Тогда $\varphi$ инъективна $\iff$ $\varphi$ сюръективна\\
    что-то про принцип дирихле и готово)
\end{proof}

\end{normalsize}


