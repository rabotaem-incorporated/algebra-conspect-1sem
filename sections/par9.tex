\begin{normalsize}

\section{Основная теорема арифметики}

\begin{theorem}
    $\letus n \geq 2$. Тогда $n$ можно представить в виде произведения простых чисел, и такое представление единственно с точностью до порядка сомножителей.    
\end{theorem}

\begin{proof}\\
    Существование:\\
    $\letus n_0$ --- наименьшее число $(\geq 2)$, для которого такого представления нету.\\
    $n_0$ --- составное число $\implies n_0 = ab, 2 \leq a, b < n_0$\\
    По минимальности $\implies a = p_1 \ldots p_k, b = q_1 \ldots q_l$, все $p_i, q_j$ --- простые.\\
    $\implies n_0 = p_1 \ldots p_k q_1 \ldots q_l$ --- Противоречие\\
    Единственность:\\
    $n = p_1 \ldots p_k = q_1 \ldots q_l$, $p_i, q_j$ --- простые.\\
    Нудно доказать: $k = l,~q_1, \ldots, q_k$ совпадают с $p_1, \ldots, p_k$ с точностью до порядка.\\
    Не умаляя общности иожно считать: $k \leq l$. \\
    Индукция по $k$:\\
    $k = 1$: $p_1 = q_1 \ldots q_l$, $p_1$ --- простое $\implies l = 1, p_1 = q_1$\\
    k > 1: $p_k \mid n \implies p_k \mid (q_1 \ldots q_l) \implies \exists j : p_k \mid q_j \implies p_k \sim q_j \implies p_k = q_j \implies$\\
    $p_1 \ldots p_{k-1} = q_1 \ldots \hat{q_j} \ldots q_l,~k-1 \leq l-1$\\
    $k-1 < k \implies$ применим индукционный переход:\\
    $k-1 = l-1$ и $q_1, \ldots, \hat{q_j}, \ldots, q_k$ --- это $p_1, \ldots, p_{k-1}$ с точностью до порядка. $\implies$\\
    $q_1, \ldots, (q_j = p_k), \ldots, q_k$ --- это $p_1, \ldots, p_k$ с точностью до порядка.
\end{proof}

\begin{defn}
    Каноническое разложение(факторизация) числа $n$ --- это представление $n$ в виде $p_1^{r_1} \ldots p_s^{r_s}$, где $p_i$ --- разложение $n$ на простые множители, $r_i \in \mathN$
\end{defn}

\begin{examples}
    \begin{itemize}
        \item $n = 112 = 2^4 \cdot 7$
        \item $n = 6006 = 2^1 \cdot 3^1 \cdot 7^1 \cdot 11^1 \cdot 13^1$
    \end{itemize}
\end{examples}

\begin{theorem-non}
    $\letus a = p_1^{r_1} \ldots p_s^{r_s}, b = p_1^{t_1} \ldots p_s^{t_s}$\\
    Тогда $a \mid b \iff r_i \leq t_i~\forall i \in \{1, \ldots, s\}$
\end{theorem-non}

\begin{proof}
    "$\Rightarrow$":\\
    $b = a \cdot p_1^{t_1 - r_1} \ldots p_s^{t_s - r_s}$\\
    "$\Leftarrow$":\\
    $b = ac$ $c = p_1^{m_1} \ldots p_s^{m_s} p_{s+1}^{m_{s+1}} \ldots p_n^{m_n}$\\
    $p_1^{t_1} \ldots p_s^{t_s} = p_1^{r_1 + m_1} \ldots p_s^{r_s + m_s} p_{s+1}^{m_{s+1}} \ldots p_n^{m_n} \implies$\\
    $t_i = r_i + m_i~\forall i \in \{1, \ldots, s\}, m_{s+1} = \ldots = m_n = 0 \implies t_i \geq r_i~\forall i \in \{1, \ldots, s\}$
\end{proof}

\begin{follow}
    $\letus a = p_1^{r_1} \ldots p_s^{r_s}$\\
    Тогда $\{d > 0 : d \mid a\} = \{p_1^{t_1} \ldots p_s^{t_s} \mid 0 \leq t_i \leq r_i,~\forall i \in \{1, \ldots, s\}\}$
\end{follow}

\begin{follow}
    $\letus a = p_1^{r_1} \ldots p_s^{r_s},~b = p_1^{t_1} \ldots p_s^{t_s}$\\
    Тогда $\gcd(a, b) = p_1^{min(r_1, t_1)} \ldots p_s^{min(r_s, t_s)}$
\end{follow}

\begin{defn}
    $\letus a, b \in \mathZ$. Число $c \in \mathZ$ называется наименьшим общим кратным чисел $a$ и $b$, если
    \begin{enumerate}
        \item $a \mid c, b \mid c$
        \item Если $a \mid c', b \mid c'$, то $c \mid c'$
    \end{enumerate}
\end{defn}

\begin{theorem-non}
    $\letus a = p_1^{r_1} \ldots p_s^{r_s}, b = p_1^{t_1} \ldots p_s^{t_s}$\\
    Тогда $c = p_1^{max(r_1, t_1)} \ldots p_s^{max(r_s, t_s)}$ --- наименьшее общее кратное чисел $a$ и $b$
\end{theorem-non}

\begin{proof}
    $a \mid c, b \mid c$ --- очевидно\\
    $\letus a \mid c', b \mid c', c' = p_1^{m_1} \ldots p_s^{m_s} p_{s+1}^{m_{s+1}} \ldots p_n^{m_n}$\\
    $a \mid c',~b \mid c' \implies r_i \leq m_i,~t_i \leq m_i,~\forall i \in \{1, \ldots, s\} \implies$\\
    $max(r_i, t_i) \leq m_i,~\forall i \in \{1, \ldots, s\} \implies c \mid c'$
\end{proof}

\begin{defn}
    НОК($a, b$) = $lcm(a, b)$ --- положительное значение наименьшего общего кратного чисел $a$ и $b$. 
\end{defn}

\begin{follow}
    $\letus a, b \in \mathN$\\
    Тогда $lcm(a, b) \cdot \gcd(a, b) = ab$
\end{follow}

\begin{proof}
    $min(r_i, t_i) + max(r_i, t_i) = r_i + t_i$
\end{proof}

\end{normalsize}