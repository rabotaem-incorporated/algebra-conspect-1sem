\subsection{Тригонометрическая форма комплексного числа}

\begin{defn}
    
    $a + bi = r(\cos\varphi + i\sin\varphi)$
    
    $a = r\cos\varphi$
    
    $b = r\sin\varphi$
\end{defn}

\begin{defn}
    Модулем комплексного числа $z = a + bi \in \CC$ назовем:
    
    $|z| = \sqrt{a^2 + b^2}$
\end{defn}

\begin{theorem-non}~
    \begin{enumerate}
        \item $|z| \geq 0,~|z| = 0 \iff z = 0$
        
        \item $|z_1 z_2| = |z_1||z_2|$
        
        \item $|z_1 + z_2| \leq |z_1| + |z_2|$
        
        \item $|\oln{z}| = |z|$
        
        \item $z \oln{z} = |z|^2$
    \end{enumerate}
\end{theorem-non}

\begin{proof}
    \begin{enumerate}
        \item очевидно
        
        \item $z_1 = a_1 + b_1i,~z_2 = a_2 + b_2i$
        
        $|z_1 z_2|^2 = (a_1 a_2 - b_1 b_2)^2 + (a_1 b_2 + a_2 b_1)^2 = a_1^2 a_2^2 + b_1^2 b_2^2 + a_1^2 b_2^2 + a_2^2 b_1^2 = $
        
        $(a_1^2 + b_1^2)(a_2^2 + b_2^2) = |z_1|^2|z_2|^2$
        
        \item $\iff |z_1 + z_2|^2 \leq (|z_1| + |z_2|)^2$
        
        $\iff (a_1 + a_2)^2 + (b_1 + b_2)^2 \leq a_1^2 + b_1^2 + a_2^2 + b_2^2 + 2|z_1||z_2|$
        
        $\iff a_1a_2 + b_1b_2 \leq \sqrt{(a_1^2 + b_1^2)(a_2^2 + b_2^2)}$
        
        $\Leftarrow |a_1 a_2 + b_1 b_2| \leq \sqrt{(a_1^2 + b_1^2)(a_2^2 + b_2^2)}$
        
        $\iff a_1^2 a_2^2 + b_1^2 b_2^2 + 2a_1a_2b_1b_2 \leq (a_1^2 + b_1^2)(a_2^2 + b_2^2)$
        
        $\iff 2a_1a_2b_1b_2 \leq b_1^2 a_2^2 + a_1^2 b_2^2$
        
        $\iff (b_1 a_2 - b_2 a_1)^2 \geq 0$

        \item очевидно
        
        \item $z = a + bi \implies \oln{z} = a - bi$
        
        $z \oln{z} = (a + bi)(a - bi) = a^2 - (bi)^2 = a^2 + b^2 = |z|^2$
    \end{enumerate}
\end{proof}

\begin{notice}
    $z^{-1} = \frac{\oln{z}}{|z|^2} = \frac{a}{a^2 + b^2} - i\frac{b}{a^2 + b^2}$
\end{notice}

\begin{defn}

    Пусть $z \in \CC$. Аргументом $z$ назовем такое $\varphi \in \R$, 

    что $z = |z|(\cos\varphi + i\sin\varphi)$
\end{defn}

\begin{theorem-non}~
    \begin{enumerate}
        \item Если $z = 0$, то любой $\varphi \in \R$ - аргумент $z$
        
        \item Если $z \neq 0$, то:
        \begin{enumerate}
            \item аргумент существует
            
            \item если $\varphi_0$ - аргумент $z$, то $\varphi$ - аргумент $z \iff \varphi = \varphi_0 + 2 \pi k,~k \in \Z$
        \end{enumerate}
    \end{enumerate}
\end{theorem-non}

\begin{proof}
    \begin{enumerate}
        \item тривиально
        
        \item $z_0 = \frac{1}{|z|} \cdot z$ 
        
        $|z_0| = \left| \frac{1}{|z|} \right| \cdot |z| = \frac{1}{|z|} \cdot |z| = 1$
        
        $z_0 = a_0 + ib_0,~|z_0| = a_0^2 + b_0^2 = 1 \implies \exists \varphi_0:
        \begin{cases}
            a_0 = \cos\varphi_0\\
            b_0 = \sin\varphi_0
        \end{cases}$

        $z = |z| \cdot z_0 = |z|(\cos\varphi_0 + i\sin\varphi_0)$

        $\varphi = \varphi_0 + 2 \pi k \implies
        \begin{cases}
            \cos\varphi = \cos\varphi_0\\
            \sin\varphi = \sin\varphi_0
        \end{cases} \implies \varphi$ - аргумент

        $\varphi$ - аргумент $\implies z = |z|(\cos\varphi + i\sin\varphi) \implies
        \begin{cases}
            \cos\varphi = \cos\varphi_0\\
            \sin\varphi = \sin\varphi_0
        \end{cases} \implies \varphi - \varphi_0 = 2 \pi k,~k \in \Z$
    \end{enumerate}
\end{proof}

\begin{defn}
    $arg(z) = \varphi$ означает $\varphi$ - один из аргументов $z$
\end{defn}

\begin{notice}
    Предположим оказалось, что $z = r(\cos\varphi + i\sin\varphi)$ для некоторых $r \geq 0,~\varphi \in \R$.\\
    Тогда $r = |z|,~\varphi = \arg z$
\end{notice}

\begin{proof}
    $|z| = \sqrt{(r\cos\varphi)^2 + (r\sin\varphi)^2} = \sqrt{r^2} = r \implies \varphi$ - аргумент по определению\\
\end{proof}

\begin{theorem-non}~
    \begin{enumerate}
        \item $\arg \oln{z} = -\arg z$
        \item $z \in \R \iff \arg z = k \pi,~k \in \Z$
        \item $\arg(z_1 z_2) = \arg z_1 + \arg z_2$
        \item $\letus z_2 \neq 0 \implies \arg \frac{z_1}{z_2} = \arg z_1 - \arg z_2$
    \end{enumerate}
\end{theorem-non}

\begin{proof}
    \begin{enumerate}
        \item $\arg z = \varphi$
        
        $z = |z|(\cos\varphi + i\sin\varphi)$
        
        $\oln{z} = |z|(\cos\varphi - i\sin\varphi) = |\oln{z}|(\cos(-\varphi) + i\sin(-\varphi)) \implies$
        
        $\arg \oln{z} = -\varphi$

        \item "$\Rightarrow$":
        
        $z > 0$: 
        
        $z = |z| \cdot 1 = |z|(\cos 0 + i\sin 0) \implies \arg z = 0$
        
        $z < 0$: 
        
        $z = |z| \cdot (-1) = |z|(\cos \pi + i\sin \pi) \implies \arg z = \pi$
        
        "$\Leftarrow$":
        
        $\sin(k \pi) = 0$

        \item $\arg z_1 = \varphi_1,~\arg z_2 = \varphi_2 \implies$
        
        $(!)~\varphi_1 + \varphi_2 = \arg(z_1 z_2)$
        
        $z_1 = |z_1|(\cos\varphi_1 + i\sin\varphi_1),~z_2 = |z_2|(\cos\varphi_2 + i\sin\varphi_2) \implies$
        
        $z_1 z_2 = |z_1| \cdot |z_2|(\cos\varphi_1 \cdot \cos\varphi_2 - \sin\varphi_1 \cdot \sin\varphi_2 + i(\sin\varphi_1 \cdot \cos\varphi_2 + \cos\varphi_1 \cdot \sin\varphi_2)) =$
        
        $|z_1 z_2|(\cos(\varphi_1 + \varphi_2) + i\sin(\varphi_1 + \varphi_2)) \implies \arg(z_1 z_2) = \varphi_1 + \varphi_2$

        \item $z_1 = \frac{z_1}{z_2} \cdot z_2 \implies \arg z_1 = \arg \frac{z_1}{z_2} + \arg z_2 \implies \arg \frac{z_1}{z_2} = \arg z_1 - \arg z_2$

    \end{enumerate}
\end{proof}

\begin{follow}(Формула Муавра)

    Пусть $z \in \CC,~|z| = r,~\arg z = \varphi,~n \in \Z$.

    Тогда $z^n = r^n(\cos(n\varphi) + i\sin(n\varphi))$
\end{follow}

\begin{proof}

    $n > 0$ --- индукция по $n$

    \textsl{База}: $n = 1$ --- тривиально
    
    \textsl{Переход}: $n - 1 \to n$
    
    $z^n = z^{n-1} \cdot z = r^{n-1}(\cos((n-1)\varphi) + i\sin((n-1)\varphi)) \cdot z = $
    
    $r^{n-1}(\cos((n-1)\varphi) + i\sin((n-1)\varphi)) \cdot r(\cos\varphi + i\sin\varphi) = $
    
    $r^n(\cos((n-1)\varphi + \varphi) + i\sin((n-1)\varphi + \varphi)) = $
    
    $r^n(\cos(n\varphi) + i\sin(n\varphi))$

    $n = 0,~ 1 = r^0(\cos(0) + i\sin(0)) = 1$

    $n < 0$: $n = -k,~k \in \N$

    $z^n = \frac{1}{z^k}$
    
    $|z^n| = \frac{1}{|z^k|} = \frac{1}{|z|^k} = |z|^{-k} = |z|^n$
    
    $\arg z^n = \arg 1 - \arg z^k = 0 - k \varphi = n \varphi$
\end{proof}