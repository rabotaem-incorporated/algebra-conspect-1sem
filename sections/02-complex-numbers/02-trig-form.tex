\subsection{Тригонометрическая форма комплексного числа}

\begin{defn}
    
    $a + bi = r(\cos\phi + i\sin\phi)$
    
    $a = r\cos\phi$
    
    $b = r\sin\phi$
\end{defn}

\begin{defn}
    Модулем комплексного числа $z = a + bi \in \CC$ назовем:
    
    $|z| = \sqrt{a^2 + b^2}$
\end{defn}

\begin{theorem-non}~
    \begin{enumerate}
        \item $|z| \geq 0,~|z| = 0 \iff z = 0$
        
        \item $|z_1 z_2| = |z_1||z_2|$
        
        \item $|z_1 + z_2| \leq |z_1| + |z_2|$
        
        \item $|\oln{z}| = |z|$
        
        \item $z \oln{z} = |z|^2$
    \end{enumerate}
\end{theorem-non}

\begin{proof}
    \begin{enumerate}
        \item очевидно
        
        \item $z_1 = a_1 + b_1i,~z_2 = a_2 + b_2i$
        
        $|z_1 z_2|^2 = (a_1 a_2 - b_1 b_2)^2 + (a_1 b_2 + a_2 b_1)^2 = a_1^2 a_2^2 + b_1^2 b_2^2 + a_1^2 b_2^2 + a_2^2 b_1^2 = $
        
        $(a_1^2 + b_1^2)(a_2^2 + b_2^2) = |z_1|^2|z_2|^2$
        
        \item $\iff |z_1 + z_2|^2 \leq (|z_1| + |z_2|)^2$
        
        $\iff (a_1 + a_2)^2 + (b_1 + b_2)^2 \leq a_1^2 + b_1^2 + a_2^2 + b_2^2 + 2|z_1||z_2|$
        
        $\iff a_1a_2 + b_1b_2 \leq \sqrt{(a_1^2 + b_1^2)(a_2^2 + b_2^2)}$
        
        $\Leftarrow |a_1 a_2 + b_1 b_2| \leq \sqrt{(a_1^2 + b_1^2)(a_2^2 + b_2^2)}$
        
        $\iff a_1^2 a_2^2 + b_1^2 b_2^2 + 2a_1a_2b_1b_2 \leq (a_1^2 + b_1^2)(a_2^2 + b_2^2)$
        
        $\iff 2a_1a_2b_1b_2 \leq b_1^2 a_2^2 + a_1^2 b_2^2$
        
        $\iff (b_1 a_2 - b_2 a_1)^2 \geq 0$

        \item очевидно
        
        \item $z = a + bi \implies \oln{z} = a - bi$
        
        $z \oln{z} = (a + bi)(a - bi) = a^2 - (bi)^2 = a^2 + b^2 = |z|^2$
    \end{enumerate}
\end{proof}

\begin{notice}
    $z^{-1} = \frac{\oln{z}}{|z|^2} = \frac{a}{a^2 + b^2} - i\frac{b}{a^2 + b^2}$
\end{notice}

\begin{defn}

    Пусть $z \in \CC$. Аргументом $z$ назовем такое $\phi \in \R$, 

    что $z = |z|(\cos\phi + i\sin\phi)$
\end{defn}

\begin{theorem-non}~
    \begin{enumerate}
        \item Если $z = 0$, то любой $\phi \in \R$ - аргумент $z$
        
        \item Если $z \neq 0$, то:
        \begin{enumerate}
            \item аргумент существует
            
            \item если $\phi_0$ - аргумент $z$, и $\phi$ - аргумент $z \iff \phi = \phi_0 + 2 \pi k,~k \in \Z$
        \end{enumerate}
    \end{enumerate}
\end{theorem-non}

\begin{proof}
    \begin{enumerate}
        \item тривиально
        
        \item $z_0 = \frac{1}{|z|} \cdot z$ 
        
        $|z_0| = \left| \frac{1}{|z|} \right| \cdot |z| = \frac{1}{|z|} \cdot |z| = 1$
        
        $z_0 = a_0 + ib_0,~|z_0| = a_0^2 + b_0^2 = 1 \implies \exists \phi_0:
        \begin{cases}
            a_0 = \cos\phi_0\\
            b_0 = \sin\phi_0
        \end{cases}$

        $z = |z| \cdot z_0 = |z|(\cos\phi_0 + i\sin\phi_0)$

       <<$\Leftarrow$>>: 

        $\phi = \phi_0 + 2 \pi k \implies
        \begin{cases}
            \cos\phi = \cos\phi_0\\
            \sin\phi = \sin\phi_0
        \end{cases} \implies \phi$ --- аргумент

        <<$\Rightarrow$>>:

        $\phi$ --- аргумент $\implies z = |z|(\cos\phi + i\sin\phi) \implies
        \begin{cases}
            \cos\phi = \cos\phi_0\\
            \sin\phi = \sin\phi_0
        \end{cases} \implies \phi - \phi_0 = 2 \pi k,~k \in \Z$
    \end{enumerate}
\end{proof}

\begin{defn}
    $arg(z) = \phi$ означает $\phi$ --- один из аргументов $z$
\end{defn}

\begin{notice}
    Предположим оказалось, что $z = r(\cos\phi + i\sin\phi)$ для некоторых $r \geq 0,~\phi \in \R$.
    Тогда $r = |z|,~\phi = \arg z$
\end{notice}

\begin{proof}
    $|z| = \sqrt{(r\cos\phi)^2 + (r\sin\phi)^2} = \sqrt{r^2} = r$, а $\phi$ --- аргумент по определению
\end{proof}

\begin{theorem-non}~
    \begin{enumerate}
        \item $\arg \oln{z} = -\arg z$
        \item $z \in \R \iff \arg z = k \pi,~k \in \Z$
        \item $\arg(z_1 z_2) = \arg z_1 + \arg z_2$
        \item $\letus z_2 \neq 0 \implies \arg \frac{z_1}{z_2} = \arg z_1 - \arg z_2$
    \end{enumerate}
\end{theorem-non}

\begin{proof}
    \begin{enumerate}
        \item $\arg z = \phi$
        
        $z = |z|(\cos\phi + i\sin\phi)$
        
        $\oln{z} = |z|(\cos\phi - i\sin\phi) = |\oln{z}|(\cos(-\phi) + i\sin(-\phi)) \implies$
        
        $\arg \oln{z} = -\phi$

        \item 
        <<$\Rightarrow$>>:
        
        $z > 0$: 
        
        $z = |z| \cdot 1 = |z|(\cos 0 + i\sin 0) \implies \arg z = 0$
        
        $z < 0$: 
        
        $z = |z| \cdot (-1) = |z|(\cos \pi + i\sin \pi) \implies \arg z = \pi$
        
        <<$\Leftarrow$>>:
        
        $\sin(k \pi) = 0$

        \item $\arg z_1 = \phi_1,~\arg z_2 = \phi_2 \implies$
        
        $\prove \phi_1 + \phi_2 = \arg(z_1 z_2)$
        
        $z_1 = |z_1|(\cos\phi_1 + i\sin\phi_1),~ z_2 = |z_2|(\cos\phi_2 + i\sin\phi_2) \implies$
        
        $z_1 z_2 = |z_1| \cdot |z_2|(\cos\phi_1 \cdot \cos\phi_2 - \sin\phi_1 \cdot \sin\phi_2 + i(\sin\phi_1 \cdot \cos\phi_2 + \cos\phi_1 \cdot \sin\phi_2)) =$
        
        $|z_1 z_2|(\cos(\phi_1 + \phi_2) + i\sin(\phi_1 + \phi_2)) \implies \arg(z_1 z_2) = \phi_1 + \phi_2$

        \item $z_1 = \frac{z_1}{z_2} \cdot z_2 \implies \arg z_1 = \arg \frac{z_1}{z_2} + \arg z_2 \implies \arg \frac{z_1}{z_2} = \arg z_1 - \arg z_2$

    \end{enumerate}
\end{proof}

\begin{follow}[Формула Муавра]

    Пусть $z \in \CC,~|z| = r,~\arg z = \phi,~n \in \Z$.

    Тогда $z^n = r^n(\cos(n\phi) + i\sin(n\phi))$
\end{follow}

\begin{proof}

    \begin{itemize}
        \item $n > 0:$ индукция по $n$

        <<База>>: $n = 1$ --- тривиально
        
        <<Переход>>: $n - 1 \to n$
        
        $z^n = z^{n-1} \cdot z = r^{n-1}(\cos((n-1)\phi) + i\sin((n-1)\phi)) \cdot z = $
        
        $r^{n-1}(\cos((n-1)\phi) + i\sin((n-1)\phi)) \cdot r(\cos\phi + i\sin\phi) = $
        
        $r^n(\cos((n-1)\phi + \phi) + i\sin((n-1)\phi + \phi)) = $
        
        $r^n(\cos(n\phi) + i\sin(n\phi))$
    
        \item $n = 0: 1 = r^0(\cos(0) + i\sin(0)) = 1$
    
        \item $n < 0:$ $n = -k,~k \in \N$
    
        $z^n = \frac{1}{z^k}$
        
        $|z^n| = \frac{1}{|z^k|} = \frac{1}{|z|^k} = |z|^{-k} = |z|^n$
        
        $\arg z^n = \arg 1 - \arg z^k = 0 - k \phi = n \phi$
    \end{itemize}    
\end{proof}