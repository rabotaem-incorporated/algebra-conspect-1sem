\subsection{Построение поля комплексных чисел}

\begin{defn}~
    $\CC = \R \times \R = \{(a, b) \mid a, b \in \R\}$
\end{defn}

\begin{defn} \
    \begin{itemize}
        \item Сложение на $\CC$: $(a_1, b_1) + (a_2, b_2) = (a_1 + a_2, b_1 + b_2)$
        
        \item Умножение на $\CC$: $(a_1, b_1) \cdot (a_2, b_2) = (a_1a_2 - b_1b_2, a_1b_2 + a_2b_1)$
    \end{itemize}
\end{defn}

\begin{theorem-non}
    $(\CC, +, \cdot)$ --- поле.
\end{theorem-non}

\begin{proof}
    \begin{itemize}
        \item Коммутативность сложения:
        
        $(a, a') + (b, b') = (a + b, a' + b') = (b + a, b' + a') = (b, b') + (a, a')$
        
        \item Ассоциативность сложения:
        
        $((a, a') + (b, b')) + (c, c') = (a + b + c, a' + b' + c') = (a + (b + c), a' + (b' + c')) = (a, a') + ((b, b') + (c, c'))$
        
        \item $(0, 0)$ --- нейтральный элемент сложения.
        
        \item $(-a, -b)$ --- обратный элемент к $(a, b)$.
        
        \item Коммутативность умножения:
        
        $(a, a') \cdot (b, b') = (ab - a'b', ab' + a'b) = (ba - b'a', ba' + b'a) = (b, b') \cdot (a, a')$
        
        \item Ассоциативность умножения: 
        
        $((a, a') \cdot (b, b')) \cdot (c, c') = (ab - a'b', ab' + a'b) \cdot (c, c') = (c(ab - a'b') - c'(ab' + a'b), c(ab' + a'b) + c'(ab - a'b')) = (cab - ca'b' - c'ab' - c'a'b, cab' + ca'b + c'ab - c'a'b') = (abc - a'b'c - ab'c' - a'bc', abc' + ab'c + abc' - a'b'c')$

        $(a, a') \cdot ((b, b') \cdot (c, c')) = (a, a') \cdot (bc - b'c', bc' + b'c) = (a(bc - b'c') - a'(bc' + b'c), a(bc' + b'c) + a'(bc - b'c')) = (abc - a'b'c - ab'c' - a'bc', abc' + ab'c + abc' - a'b'c')$

        Как видно, выражения совпадают.
        
        \item Дистрибутивность:
        
        Проверим правую, левая проверяется аналогично:

        $((a, a') + (b, b')) \cdot (c, c') = (a + b, a' + b') \cdot (c, c') = ((ac + bc) - (a'c' + b'c'), (a'c + b'c) + (ac' + bc')) = (ac - a'c', a'c + ac') + (bc - b'c', b'c + bc') = (a, a') \cdot (c, c') + (b, b') \cdot (c, c')$
        
        \item $(1, 0)$ --- нейтральный элемент умножения.
        
        \item $(a, b) z_1 z_2 = (1, 0):~z_1 = (a, -b),~z_2=\frac{1}{a^2+b^2}$
    \end{itemize}
\end{proof}

\begin{defn}
    $\CC$ --- поле комплексных чисел.
\end{defn}

\begin{defn}
    $c \in \CC$ --- комплексное число.
\end{defn}

\begin{theorem-non}
    
    $\R' = \{(a, 0) \mid a \in \R\}$, тогда:
    
    $\R'$ замкнуто относительно сложения, вычитания, умножения, содержит единицу, то есть является подкольцом поля $\CC \implies \R'$ --- само является кольцом относительно сложения, умножения, ограниченных на $\R'$. 
    
    Тогда существует отображение $\phi:\ \R \to \R'$, где $a \mapsto (a, 0)$, и $\phi(a)$ --- изоморфизм колец, 
    
    то есть $\phi$ --- биекция и $\begin{cases}
        \phi(a + b) = \phi(a) + \phi(b) \\
        \phi(ab) = \phi(a)\phi(b)
    \end{cases}$.

    Отождествим $(a, 0)$ с вещественным числом $a$.
\end{theorem-non}

Давайте наконец-то определим комплексное число.

$(a, 0) \cdot (0, 1) = (0, a)$
    
$(a, b) = (a, 0) + (0, b) = (a, 0) + (b, 0) \cdot (0, 1) = a + b \cdot (0, 1) = a + bi$

\begin{defn}
    $z = a + bi$ --- \emph{комплексное число}.

    $a = \Re(z)$, $b = \Im(z)$ --- \emph{действительная} и \emph{мнимая} части комплексного числа $z$.
    
    В геометрическом виде это вектор $z = (a, b)$.
\end{defn}

\begin{defn}
    Пусть $z = a + bi$ --- комплексное число, тогда $\oln{z} = a - bi$ --- \emph{сопряженное} к $z$.
\end{defn}


\subsubsection*{Отступление про отображения}

\begin{defn}
    $id_M: M \to M,~x \mapsto x$ --- тождественное отображение на $M$.
\end{defn}

\begin{defn}
    $\letus \alpha: M \to N,~\beta: N \to P$ --- отображения
    
    Тогда $\alpha \circ \beta: M \to P,~x \mapsto \alpha(\beta(x))$ --- композиция отображений.
\end{defn}

\begin{defn}
    $\letus \alpha: M \to N$ --- отображение

    Отображение $\beta: N \to M$ --- обратное к $\alpha$, если $\beta \circ \alpha = id_M$.
\end{defn}

\begin{theorem-non}
    У отображения $\alpha: M \to N$ есть обратное отображение, если и только если $\alpha$ --- биекция.
\end{theorem-non}

\begin{proof}
    
    <<$\Rightarrow$>>:
    
    Инъективность:
    
    $\beta \circ \alpha = id_M,~\alpha(x) = \alpha(y) \implies \beta(\alpha(x)) = \beta(\alpha(y)) \implies x = y$
    
    Сюръективность:
    
    $y \in N,~y = \alpha(\beta(y)) \in \Im(\alpha)$ ($\Im$ это прообраз)
    
    <<$\Leftarrow$>>:
    
    Пусть $\alpha$ --- биекция, назовем $\beta: N \to M$ --- обратным, если $\forall y \in N \alpha^{-1}(y) = \{x\},~x \in M$
    
    Положим $\beta(y) = x,~\alpha \circ \beta = id_N,~\beta \circ \alpha = id_M$
\end{proof}

\subsubsection*{Продолжение}

\begin{defn}
    Автоморфизм --- изоморфизм на себя.
\end{defn}

\begin{theorem-non}
    $\sigma: \CC \to \CC,~z \mapsto \oln{z}$ --- автоморфизм.
\end{theorem-non}

\begin{proof}
    
    $\sigma$ --- биекция, т.к. $\sigma \circ \sigma = id_{\CC}$
    
    $\sigma(z_1 + z_2) = \sigma(z_1) + \sigma(z_2)$ --- очевидно
    
    $\sigma(z_1 z_2) = \sigma(z_1) \sigma(z_2)$
    
    $\sigma(1) = 1$ --- очевидно
    
    $z_1 = a_1 + b_1i,~z_2 = a_2 + b_2i$
    
    $\sigma(z_1 z_2) = \oln{a_1 a_2 - b_1 b_2 + i(a_1 b_2 + a_2 b_1)} = a_1 a_2 - b_1 b_2 + i(a_1 b_2 + a_2 b_1)$
    
    $\sigma(z_1) \sigma(z_2) = \oln{(a_1 - i b_1) (a_2 - i b_2)} = a_1 a_2 - b_1 b_2 + i(a_1 b_2 + a_2 b_1)$
\end{proof}