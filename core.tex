\documentclass[a4paper,12pt]{article}
\usepackage[utf8]{inputenc}
\usepackage{cmap}
\usepackage[russian]{babel}
\usepackage{xspace}
\usepackage{centernot}
\usepackage{mathtools}
\usepackage[normalem]{ulem}

\usepackage[
  left=0.50in,
  right=0.50in,
  top=0.8in,
  bottom=0.7in,
  headheight=0.8in]{geometry}
%\pagenumbering{gobble}


\newcommand{\makefirstpages}{%
  \pagestyle{empty}
  \title{\CourseName}
  \clearpage\maketitle
  \thispagestyle{empty}
  \pagebreak

  \tableofcontents
  \pagebreak
  
  \pagestyle{fancy}
  \pagenumbering{arabic}
  \setcounter{page}{1}
}


\usepackage{parskip}
\usepackage{indentfirst}
\setlength\parindent{0cm}

\usepackage{fancyhdr}
\pagestyle{fancy}
\fancyhf{}
\renewcommand{\headrulewidth}{0pt}
\renewcommand{\footrulewidth}{1pt}
\fancyfoot[L]{\CourseName}
\fancyfoot[R]{\thepage}

\usepackage{import}
\usepackage{stackengine}

\usepackage[pdftex]{graphicx}
\graphicspath{ {images/} }

\renewcommand{\thesubsection}{\arabic{subsection}}

% TODO
\newcommand{\TODO}[1][]{
  \vspace{0.2em}
  \textbf{{\bf\color{red} TODO:} #1}
  \vspace{0.2em}
}


\usepackage{amsmath,amsthm,amssymb,mathtext}
\usepackage{thmtools}
\usepackage{tikz}

\newcommand{\Z}{\ensuremath{\mathbb{Z}}\xspace}
\newcommand{\N}{\ensuremath{\mathbb{N}}\xspace}
\newcommand{\R}{\ensuremath{\mathbb{R}}\xspace}
\newcommand{\Q}{\ensuremath{\mathbb{Q}}\xspace}
\newcommand{\F}{\ensuremath{\mathbb{F}}\xspace}
\newcommand{\CC}{\ensuremath{\mathbb{C}}\xspace} 
\renewcommand{\P}{\ensuremath{\mathbb{P}}\xspace} 

\DeclareRobustCommand{\divby}{%
    \mathrel{\vbox{\baselineskip.65ex\lineskiplimit0pt\hbox{.}\hbox{.}\hbox{.}}}%
}
\newcommand*{\ndivby}{\centernot\divby}
\newcommand{\equivm}[1]{\underset{#1}{\equiv}}
\newcommand{\eps}{\varepsilon}
\renewcommand{\leq}{\leqslant}
\renewcommand{\geq}{\geqslant}
\newcommand{\empysetold}{\emptyset}
\renewcommand{\emptyset}{\varnothing}
\newcommand{\oln}[1]{\overline{#1}}
\newcommand{\lmapsto}{\longmapsto}
\newcommand{\lto}{\longrightarrow}

\let\Im\relax
\let\Re\relax
\DeclareMathOperator\Im{Im}
\DeclareMathOperator\Re{Re}
\DeclareMathOperator\id{id}
\DeclareMathOperator*{\lcm}{lcm}
\newcommand\vphi{\phi}
\renewcommand\phi{\varphi}
\DeclareMathOperator{\ord}{ord}
\DeclareMathOperator{\Ker}{Ker}
\DeclareMathOperator{\Aut}{Aut}
\DeclareMathOperator{\Inn}{Inn}
\DeclareMathOperator{\Fix}{Fix}
\DeclareMathOperator{\Supp}{Supp}
\DeclareMathOperator{\sgn}{sgn}
\DeclareMathOperator{\Stab}{Stab}
\DeclareMathOperator{\Arg}{Arg}
\DeclareMathOperator{\Char}{char}
\DeclareMathOperator{\ideg}{indeg}
\DeclareMathOperator{\odeg}{outdeg}
\DeclareMathOperator{\Int}{Int}
\DeclareMathOperator{\Cl}{Cl}
\DeclareMathOperator{\codim}{codim}
\DeclareMathOperator{\Hom}{Hom}
\DeclareMathOperator{\rk}{rk}
\DeclareMathOperator{\Vol}{Vol}
\DeclareMathOperator{\Adj}{Adj}
\DeclareMathOperator{\End}{End}
\DeclareMathOperator{\Ann}{Ann}
\DeclareMathOperator{\Tr}{Tr}
\DeclareMathOperator{\grad}{grad}

% letus symbol
\def\letus{%
    \mathord{\setbox0=\hbox{$\exists$}%
             \hbox{\kern 0.125\wd0%
                   \vbox to \ht0{%
                      \hrule width 0.75\wd0%
                      \vfill%
                      \hrule width 0.75\wd0}%
                   \vrule height \ht0%
                   \kern 0.125\wd0}%
           }%
}   

\ifx\ThmSpacing\undefined
\def\ThmSpacing{9pt}
\fi

\ifx\ThmNamespace\undefined
\def\ThmNamespace{subsection}
\fi

\declaretheoremstyle[
 spaceabove=\ThmSpacing, spacebelow=\ThmSpacing,
 headfont=\slshape\bfseries,
 bodyfont=\normalfont,
 postheadspace=0.5em,
]{thmstyle_def}

\declaretheoremstyle[
 spaceabove=\ThmSpacing, spacebelow=\ThmSpacing,
 postheadspace=0.5em,
]{thmstyle_thm}

\declaretheoremstyle[
 spaceabove=\ThmSpacing, spacebelow=\ThmSpacing,
 headfont=\itshape\bfseries,
 notefont=\itshape\bfseries, notebraces={}{},
 bodyfont=\normalfont,
 postheadspace=0.5em,
]{thmstyle_cons}

\declaretheoremstyle[
 spaceabove=\ThmSpacing, spacebelow=\ThmSpacing,
 headfont=\bfseries,
 notefont=\bfseries, notebraces={}{},
 bodyfont=\normalfont,
 postheadspace=0.5em,
]{thmstyle_examp}  

\declaretheoremstyle[
 spaceabove=\ThmSpacing, spacebelow=\ThmSpacing,
 headfont=\ttfamily\itshape,
 notefont=\ttfamily\itshape, notebraces={}{},
 bodyfont=\normalfont,
 postheadspace=0.5em,
]{thmstyle_remark}

\declaretheorem[parent=\ThmNamespace, name=Теорема, style=thmstyle_thm]{theorem}
\declaretheorem[parent=\ThmNamespace, name=Предложение, style=thmstyle_thm]{theorem-non}
\declaretheorem[parent=\ThmNamespace, name=Утверждение, style=thmstyle_thm]{statement}
\declaretheorem[parent=\ThmNamespace, name=Определение, style=thmstyle_def]{defn}
\declaretheorem[parent=\ThmNamespace, name=Лемма, style=thmstyle_thm]{lemma}
\declaretheorem[numbered=no, name=Замечание, style=thmstyle_remark]{notice}
\declaretheorem[numbered=no, name=Следствие, style=thmstyle_cons]{follow}
\declaretheorem[numbered=no, name=Пример, style=thmstyle_examp]{example}
\declaretheorem[numbered=no, name=Примеры, style=thmstyle_examp]{examples}
\declaretheorem[numbered=no, name=Свойства, style=thmstyle_cons]{prop}
\declaretheorem[numbered=no, name=Свойство, style=thmstyle_cons]{property}
\declaretheorem[numbered=no, name=Упражнение, style=thmstyle_examp]{exerc}
\renewcommand{\proof}[1][]{\textbf{Доказательство#1.} } 

\usepackage{hyperref}
\hypersetup{
  colorlinks,
  citecolor=black,
  filecolor=black,
  linkcolor=blue,
  urlcolor=blue
}

\let\oldsection\section
\renewcommand\section{\clearpage\oldsection}

\newcommand{\lbl}[2]{\overset{\text{#1}}{#2}}
\newcommand{\prove}{(!)\ }

% они все равно убогие
\renewcommand{\le}{\geq}
\renewcommand{\ge}{\geq}
